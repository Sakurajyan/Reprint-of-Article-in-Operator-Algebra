% \documentclass[12pt]{report}
% \usepackage{geometry}
%  \geometry{
%  a4paper,
%  total={170mm,257mm},
%  left=20mm,
%  top=20mm,
%  }
% \usepackage[hidelinks]{hyperref}
% \usepackage{tikz-cd}
% \usepackage{amsthm}
% \usepackage{amsmath}
% \usepackage{amssymb}
% \usepackage{amsfonts}
% \usepackage{xcolor}
% \usepackage{physics}
% \usepackage{mathrsfs}
% \usepackage{bbm}

% \tikzcdset{every label/.append style = {font = \normalsize}}
% \newtheorem{theorem}{Theorem}[section]
% \newtheorem{proposition}[theorem]{Proposition}
% \newtheorem{corollary}[theorem]{Corollary}
% \newtheorem{definition}[theorem]{Definition}
% \theoremstyle{definition}
% \newtheorem{example}[theorem]{Example}
% \newcommand{\inner}[1]{\langle#1\rangle}
% \newcommand{\Sp}{\operatorname{Sp}}
% \begin{document}
% \tableofcontents
\chapter{Measurable Operators with Respect to a Trace}
In this chapter, we define the notion of measurability with respect to a trace $\tau$ on a von Neumann algebra $M$ and show that the set $M$ of $\tau$-measurable operators is a complete topological *-algebra. Our presentation is a modified version of that given by E. Nelson [13].\par
Let $M$ be a - necessarily semifinite - von Neumann algebra acting on a Hilbert space $H$ and let $\tau$ be a normal faithful semifinite trace on $M$.\par
For the convenience of the reader, we immediately give the definition of $\tau$-measurability and state the main theorem about $\tau$-measurable operators.\par
\bigskip
Definition 14: A closed densely defined operator $a$ affiliated with $M$ is called $\tau$-measurable if for all $\delta\in \mathbb{R}_+$ there exists a projection $p\in M$ such that
\[
pH\subset D(a)\text{ and } \tau(1-p)\leq \delta
\]
For a characterization of $\tau$-measurable operators in terms of the spectral projections of their absolute value, see Proposition 21 below.\par
We denote by $\widetilde{M}$ the set of $\tau$-measurable closed densely defined operators.\par
% \newpage
Theorem 28. 1) $\widetilde{M}$ is a *-algebra with respect to strong sum, strong product, and adjoint operation.\par
2) The sets
\[
    N(\epsilon,\delta)=\{a\in \widetilde{M}|\exists p\in M_{\text{proj}}:pH\subset D(a),\norm{ap}\leq \epsilon,\tau(1-p)\leq \delta\},
\]
where $\epsilon,\delta\in \mathbb{R}_+$, form a basis for the neighbourhoods of $0$ for a topology on $\widetilde{M}$ that turns $\widetilde{M}$ into a topological vector space. \par
3) $\widetilde{M}$ is a complete Hausdorff topological * -algebra and M is a dense subset of $\widetilde{M}$.\par
\bigskip
Once this theorem has been proven, we can freely add and multiply operators from $\widetilde{M}$, the operations being understood in the strong sense (see the definition below). Until then, we have to deal with unbounded operators in the usual careful way.\par
Although we are mainly interested in closed densely defined opera tors it will be convenient for us to work with more general kinds of unbounded operators. We therefore start by recalling some basic facts on arbitrary unbounded operators. Next, we recall some properties of the lattice $M_{\text{proj}}$ of projections in $M$. After this, we go on to develop the theory of $\tau$-measurability.\par
\bigskip
\section*{Preliminaries on unbounded operators.}\par
Recall that for any (linear) operators $a$ and $b$ on $H$ we can define the sum $a+b$ and the product $ab$ as operators on $H$ with domains
\begin{equation}
    D(a+b)=D(a)\cap D(b),
\end{equation}
\begin{equation}
    D(ab)=\{\xi\in D(b)|b\xi\in D(a)\}.
\end{equation}
% \newpage
These operations are associative so that $a+b+c$ and $abc$ are well-defined operators. Furthermore, for all $a$, $b$ and $c$ we have
\begin{equation}
    (a+b)c = ac + bc \text{ and } c(a+b)\supset ca + cb
\end{equation}
(with equality if $D(c)=H$).\par
We shall use the following terminology: an operator $a$ on $H$ is closed if its graph $G(a)$ is closed in $H\otimes H$; $a$ is preclosed if the closure $\overline{G(a)}$ of its graph is the graph of some - necessarily closed - operator (the closure of $a$, denoted $[a]$; $a$ is densely defined if $D(a)$ is dense in $H$.\par
If $a$, $b$ and $ab$ are densely defined, then
\begin{equation}
    (ab)^*\supset b^*a^*
\end{equation}
with equality if $a$ is bounded and everywhere defined.\par
A closed densely defined operator $a$ has a unique polar decomposition
\begin{equation}
    a = u\abs{a}
\end{equation}
where $\abs{a}$ is a positive self-adjoint operator and $u$ a partial isometry with $\supp(a)$ as its initial projection and $r(a)$, the projection onto the closure of the range of $a$, as its final projection.\par
If the sum $a+b$ of two closed densely defined operators $a$ and $b$ is preclosed and densely defined, then the closure $[a+b]$ is called the strong sum of $a$ and $b$. Similarly, the strong product is the closure $[ab]$ if $ab$ is preclosed and densely defined.\par
% \newpage
We shall write
\[
    \norm{a}=\sup\{\norm{a\xi}| \norm{\xi}\leq 1\}  
\]
for all everywhere defined operators $a$ on $H$, bounded or not.For all such operators, the usual norm estimates hold:
\[
    \norm{a+b}\leq \norm{a}+\norm{b}, \norm{ab}\leq \norm{a}\norm{b}.
\]
Denote by $M'$ the commutant of $M$.\par
\begin{definition}
    A linear operator $a$ on $H$ is said to be affiliated with $M$ (and we write $a\eta H$) if
\begin{equation}
    \forall y\in M':ya\subset ay
\end{equation}
\end{definition} 
\begin{remark}
    Using (3), (4) and (5) one easily verifies that
\end{remark}
\begin{enumerate}
    \item if $a,b\eta M$, then $a+b\eta M$ and $ab\eta H$;
    \item if $a$ is preclosed, resp. densely defined, and $a\eta M$, then $[a] \eta M$, resp. $a^*\eta M$;
    \item if $a$ is a closed densely defined operator with polar decomposition $a=u\abs{a}$, then $a\eta M$ if and only if $u\in M$.
\end{enumerate}
Notation. We denote by $\overline{M}$ the set of closed densely defined operators affiliated with $M$.\par
\bigskip
\section*{Preliminaries on projections.}
We denote by $M_{\text{proj}}$ the lattice of (orthogonal) projections in $M$. For a family $(p_i)_{i\in I}$ of projections in $M$, $\wedge_{i\in I}p_i$ (resp. $\vee_{i\in I}p_i$) is the projection onto $\cap_{i\in I}p_iH$ (resp. $\overline{\cup_{i\in I}p_iH}$).\par
Recall that
\begin{equation}
    (\wedge_{i\in I}p_i)^\perp=\vee_{i\in I}p_i^\perp, (\vee_{i\in I}p_i)^\perp=\wedge_{i\in I}p_i^\perp
\end{equation}
where $p^\perp=1-p$ is the projection orthogonal to $p$\par
Two projections $p$ and $q$ are equivalent if $p=u^*u$ and $q=uu^*$ for some $u\in M$. We denote equivalence by $\sim$. Equivalent projections have the same trace.\par
By the polar decomposition theorem, we have
\begin{lemma}
    Let $a$ be a closed densely defined operator affiliated
with $M$. Then
\[
    \supp(a)\sim \rsupp(a)
\]
where $\rsupp(a)$ denotes the projection onto the closure of the range of $a$.
\end{lemma}
For any projections $p,q\in M$ we have
\begin{equation}\label{(8)}
    (p\vee q)-p\sim q-(p\wedge q).
\end{equation}
It follows that
\begin{equation}\label{(9)}
    \tau(p\vee q)\leq \tau(p)+\tau(q).
\end{equation}
More generally,
\begin{equation}
    \tau(\vee_{i\in I}p_i)\leq \sum_{i\in I}\tau(p_i)
\end{equation}
for any family $(p_i)_{i\in I}$ of projections in $M$ (if I is finite, this follows by induction from \eqref{(9)}; for the general case, use the normality of $\tau$).\par
Another consequence of \eqref{(8)} is this:
\begin{equation}\label{(11)}
    \forall p,q\in M_{\text{proj}}: p\wedge q=0\Rightarrow p\lesssim 1-q
\end{equation}
(where $\lesssim$ means: "is equivalent to a subprojection of"). Indeed,
\[
    p=1-p^\perp=(p\wedge q)^\perp-p^\perp=(p^\perp\vee q^\perp)-p^\perp\sim q^\perp-(p^\perp \wedge q^\perp)\leq q^\perp=1-q.
\]
\section*{The theory of $\tau$-measurable operators.}
\begin{definition}
   Let $\epsilon,\delta\in \mathbb{R}_+$. Then we denote by $D(\epsilon,\delta)$ the set of all operators $a\eta M$ for which there exists a projection $p\in M$ such that
   \begin{enumerate}
       \item $pH \subset D(a)$ and $\norm{ap}\leq \epsilon$ and
       \item $\tau(1-p)\leq \delta$.
   \end{enumerate}
\end{definition}
When $pH\subset D(a)$, the operator $ap$ is everywhere defined. The requirement $\norm{ap}\leq \epsilon$ in particular implies that $ap$ is bounded.\par
Note that we do not require $a$ to be densely defined, closed or preclosed.
\begin{proposition}
    Let $\epsilon_1,\epsilon_2,\delta_1,\delta_2\in \mathbb{R}_+$. Then 
    \begin{enumerate}
        \item $D(\epsilon_1,\delta_1)+D(\epsilon_2,\delta_2)\subset D(\epsilon_1+\epsilon_2,\delta_1+\delta_2)$,
        \item $D(\epsilon_1,\delta_1)D(\epsilon_2,\delta_2)\subset D(\epsilon_1\epsilon_2,\delta_1+\delta_2)$.
    \end{enumerate}
\end{proposition}
\begin{proof}
    (1) Let $a\in D(\epsilon_1,\delta_1)$ and $b\in D(\epsilon_2,\delta_2)$. Then there exist projections $p,q\in M$ such that
    \[
      \begin{split}
          pH\subset D(a),\norm{ap}\leq \epsilon_1,\text{ and } \tau(1-p)\leq \delta_1,\\
            qH\subset D(b),\norm{bq}\leq \epsilon_2,\text{ and } \tau(1-q)\leq \delta_2.
      \end{split}  
    \]
    Put $r=p\wedge q$. Then
      \[
          rH=pH\cap qH\subset D(a)\cap D(b)=D(a+b)
      \]
      and
      \[
          \norm{(a+b)r}=\norm{ar+br}\leq\norm{ar}+\norm{br}\leq \norm{ap}+\norm{bq}\leq \epsilon_1+\epsilon_2
      \]
    Furthermore,
    \[
        \tau(1-r)=\tau((p\wedge q)^\perp)=\tau(p^\perp\vee q^\perp)\leq \tau(1-p)+\tau(1-q)\leq \delta_1+\delta_2
    \]
    This proves (1).\par
    To prove (2), let $a\in D(\epsilon_1,\delta_1)$, $b\in D(\epsilon_2,\delta_2)$ and take $p,q\in M_{\text{proj}}$ as above. Then $bq$, and hence $(1-p)bq$, is bounded. Denote by $s$ the projection onto its null space:
    \[
        sH=N((1-p)bq).
    \]
    Then $bq\xi\in pH\subset D(a)$ for all $\xi\in sH$, so that $sH\subset D(abq)$ and hence
    \[
    (q\wedge s)H\subset D(ab)      
    \]
    Also, $abqs=apbqs$ so that
    \[
        ab(q\wedge s)=abqs(q\wedge s)=apbq(q\wedge s)
    \]
    and thus 
    \[
        \norm{ab(q\wedge s)}\leq\norm{ap}\norm{bq}\leq \epsilon_1\epsilon_2. 
    \]
    On the other hand, using that
    \[
        1-s=\supp((1-p)bq)\sim \rsupp((1-p)bq)\leq 1-p,
    \]
    we have
    \[
        \begin{split}
            \tau(1-(q\wedge s))=\tau((1-q)\vee (1-s))\leq& \tau(1-q)+\tau(1-s)\\
            \leq &\tau(1-q)+\tau(1-p)\leq \delta_1+\delta_2.
        \end{split}
    \]
    This completes the proof.
\end{proof}
\begin{proposition}
    Let $\epsilon,\delta\in \mathbb{R}_+$.
    \begin{enumerate}
        \item Let $a$ be a preclosed operator. Then
        \[
            a\in D(\epsilon,\delta)\Rightarrow [a]\in D(\epsilon,\delta).
        \]
        \item Let a be a closed densely defined operator with polar decomposition $a=u\abs{a}$. Then
        \[
            a\in D(\epsilon,\delta)\Leftrightarrow u\in M\text{ and }\abs{a}\in D(\epsilon,\delta).
        \]
    \end{enumerate}
\end{proposition}
\begin{proof}
    (1): trivial. (2): trivial, since $a=u\abs{a}$, $\abs{a}=u^*a$, and $\norm{u}\leq 1$.
\end{proof}
\begin{lemma}\label{Lemma 7}
    Let $a\in\overline{M}$ and $\epsilon,\delta\in \mathbb{R}_+$. Then
    \[
        a\in D(\epsilon,\delta)\Leftrightarrow \tau(\chi_{]\epsilon,\infty[}(\abs{a}))\leq \delta
    \]
    (where $\chi_{]\epsilon,\infty[}(\abs{a})$ denotes the spectral projection of $\abs{a}$ corresponding to the interval $]\epsilon,\infty[$).
\end{lemma}
\begin{proof}
    "$\Leftarrow$": Put $p=\chi_{[0,\epsilon]}(\abs{a})$. Then $pH\subset D(\abs{a})$ and $\norm{\abs{a}p}\leq \epsilon$.\par
    "$\Rightarrow$": For some $p\in M_{\text{proj}}$, we have
    \[
        \norm{\abs{a}p}\leq \epsilon \text{ and } \tau(1-p)\leq \delta.
    \]
    Let $\abs{a}=\int_0^\infty\lambda\dd e_\lambda$ be the spectral decomposition of $\abs{a}$. Now for all $\xi\in pH$ we have
    \[
        \norm{\abs{a}\xi}^2\leq \epsilon^2\norm{\xi}^2,
    \]
    and for all $\xi\in (1-e_{\epsilon})H\setminus\{0\}$ we have
    \[
        \norm{\abs{a}\xi}^2>\epsilon^2\norm{\xi}^2
    \]
    since
    \[
        \norm{\abs{a}\xi}^2=\int_0^\infty\lambda^2\dd (e_\lambda \xi|\xi)=\int_{]\epsilon,\infty[}\lambda^2\dd(e_\lambda\xi|\xi).
    \]
    Hence $(1-e_\epsilon)H\cap pH$ must be $\{0\}$, i.e. $(1-e_\epsilon)\wedge p=0$. By \eqref{(11)} we conclude that $1-e_\epsilon\lesssim 1 - p$, whence $\tau(1-e_\epsilon)\leq \delta$.
\end{proof}
\begin{proposition}
    Let $a\in \overline{M}$ and $\epsilon,\delta\in \mathbb{R}_+$. Then
    \[
        a\in D(\epsilon,\delta)\Leftrightarrow a^*\in D(\epsilon,\delta)  
    \]
\end{proposition}
\begin{proof}
    Let $a=u\abs{a}$ be the polar decomposition of $a$. Then $u$ is an isometry of $\chi_{]0,\infty[}(\abs{a})=\supp(a)$ onto $\chi_{]0,\infty[}(\abs{a^*})=\supp(a^*)=\rsupp(a)$. By uniqueness of the spectral decomposition, $u$ induces for each $\lambda\in \mathbb{R}_+$ an isometry of $\chi_{]\lambda,\infty[}(\abs{a})$ onto $\chi_{]\lambda,\infty[}(\abs{a^*})$.
    The result follows by Lemma \ref{Lemma 7}.
\end{proof}
\begin{definition}
    A subspace $E$ of $H$ is called $\tau$-dense if for all $\delta\in \mathbb{R}_+$, there exists a projection $p\in M$ such that
    \[
        pH\subset E \text{ and } \tau(1-p)\leq \delta.  
    \]
\end{definition}
\begin{proposition}
    Let $E$ be a $\tau$-dense subspace of $H$ . Then there exists an increasing sequence $(p_n)_{n\in \mathbb{N}}$ of projections in $M$ with
\[
    p_n\nearrow 1,\tau(1-p_n)\to 0,\text{ and } \cup_{n=1}^\infty p_nH\subset E.
\]
\end{proposition}
\begin{proof}
    Take projections $q_k\in M$, $k\in N$, such that 
    \[
        q_kH\subset E \text{ and } \tau(1-q_k)\leq 2^{-k}.
    \]   
    For each $n\in N$, put
    \[
        p_n=\wedge_{k=n+1}^\infty q_k.    
    \]
    Then
    \[
        p_nH=\cap_{k=n+1}^\infty q_kH\subset E  
    \]
    and 
    \[
        \tau(1-p_n)=\tau\left( \vee_{k=n+1}^\infty(1-q_k) \right)\leq \sum_{k=n+1}^\infty\tau(1-q_k)\leq \sum_{k=n+1}^\infty 2^{-k}=2^{-n}  
    \]
    It follows that
    \[
        p_n\nearrow 1;  
    \]
    indeed, denoting by $p$ the supremum of the increasing sequence $p_n$, we have
    \[
        \forall n\in \mathbb{N}:\tau(1-p)\leq \tau(1-p_n)\leq 2^{-n} 
    \]
    whence $\tau(1-p)=0$ and $p=1$.\par
    Furthermore,
    \[
        \cup_{n=1}^\infty p_nH\subset E.  
    \]
\end{proof}
\begin{corollary}
    Let $E$ be a $\tau$-dense subspace of $H$. Then $E$ is dense in $H$.
\end{corollary}
An important property of $\tau$-dense subspaces is the following:
% \end{document}