% \documentclass[12pt]{report}
% \usepackage{geometry}
%  \geometry{
%  a4paper,
%  total={170mm,257mm},
%  left=20mm,
%  top=20mm,
%  }
% \usepackage[hidelinks]{hyperref}
% \usepackage{tikz-cd}
% \usepackage{amsthm}
% \usepackage{amsmath}
% \usepackage{amssymb}
% \usepackage{amsfonts}
% \usepackage{xcolor}
% \usepackage{physics}
% \usepackage{mathrsfs}
% \usepackage{bbm}

% \tikzcdset{every label/.append style = {font = \normalsize}}
% \newtheorem{theorem}{Theorem}[section]
% \newtheorem{proposition}[theorem]{Proposition}
% \newtheorem{corollary}[theorem]{Corollary}
% \newtheorem{definition}[theorem]{Definition}
% \theoremstyle{definition}
% \newtheorem{example}[theorem]{Example}
% \newcommand{\inner}[1]{\langle#1\rangle}
% \newcommand{\Sp}{\operatorname{Sp}}
% \begin{document}
% \tableofcontents
\chapter{Spatial Derivatives}
Spatial derivatives were introduced by A. Connes in [1]. In this chapter, we give an alternative definition (equivalent to that given in [1]) suggested to us by U. Haagerup, based on the notion of the extended positive part of a von Neumann algebra. This definition permits us to obtain very easily some elementary properties of spatial derivatives. After this, we recall their main modular properties and the characterization as (-1)-homogeneous operators.\par
\section{Definition and elementary properties of spatial derivatives}
Let $M$ be a von Neumann algebra acting on a Hilbert space $H$, and let $\psi$ be a normal faithful semifinite weight on the commutant $M'$ of $M$.\par
We shall use the following standard notation: $n_\psi=\{y\in M'|\psi(y^*y)<\infty\}$, $H_\psi$ the Hilbert space completion of $n_\psi$ with respect to the inner product $(y_1,y_2)\mapsto \psi(y_2^*y_1)$, $\Lambda_\psi$ the canonical injection of $n_\psi$ into $H_\psi$, $\pi_\psi$ the canonical representation of $M'$ on $H_\psi$.
\begin{definition}\label{Chap3: Def: 1}
    For each $\xi\in H$, we denote by $R^\psi(\xi)$ the (densely defined) operator from $H_\psi$ to $H$ defined by
    \begin{equation}
        R^\psi(\xi)\Lambda_\psi(y)=y\xi,y\in n_\psi.
    \end{equation}
\end{definition}
\begin{proposition}\label{Chap3: Prop: 2}
    For all $\xi,\xi_1,\xi_2\in H$, $x\in M$, and $y\in M'$ we have
    \begin{enumerate}[(i)]
        \item $R^\psi(\xi_1+\xi_2)=R^\psi(\xi_1)+R^\psi(\xi_2)$,
        \item $R^\psi(x\xi)=xR^\psi(\xi)$,
        \item $yR^\psi(\xi)\subset R^\psi(\xi)\pi_\psi(y)$,
    \end{enumerate}
    and
    \begin{enumerate}[(i)*]
        \item $R^\psi(\xi_1)^*+R^\psi(\xi_2)^*\subset R^\psi(\xi_1+\xi_2)^*$,
        \item $R^\psi(x\xi)^*=R^\psi(\xi)^*x^*$,
        \item $\pi_\psi(y)R^\psi(\xi)^*\subset R^\psi(\xi)^*y$.
    \end{enumerate}
\end{proposition}
\begin{proof}
    (i) and (ii) are immediate from Definition \ref{Chap3: Def: 1}. (iii): For all $z\in n_\psi$, we have $yR^\psi(\xi)\Lambda_\psi(z)=yz\xi=R^\psi(\xi)\Lambda_\psi(yz)=R^\psi(\xi)\pi_\psi(y)\Lambda_\psi(z)$.\par
    (i)*, (ii)*, and (iii)* follow from (i), (ii), and (iii) using $R^\psi(\xi_1)+R^\psi(\xi_2)\subset (R^\psi(\xi_1)+R^\psi(\xi_2))^*$, $(xR^\psi(\xi))^*=R^\psi(\xi)^*x^*$, and $(y^*R^\psi(\xi))^*=R^\psi(\xi)^*y^*$.
\end{proof}
\begin{definition}
    A vector $\xi\in H$ is called $\psi$-bounded if the operator $R^\psi(\xi)$ is bounded. The set of $\psi$-bounded vectors is denoted $D(H,\psi)$.
\end{definition}
\begin{notation}
    If $\xi\in D(H,\psi)$, $R^\psi(\xi)$ extends to a bounded operator $H_\psi\to H$ which we shall also denote $R^\psi(\xi)$.
\end{notation}
\begin{proposition}
    The set $D(H,\psi)$ is an $M$-invariant dense subspace
    of $H$.
\end{proposition}
\begin{proof}
    That $D(H,\psi)$ is an $M$-invariant subspace of $H$ follows from Proposition \ref{Chap3: Prop: 2}, i) and (ii). Denote by $e$ the projection onto $\overline{D(H,\psi)}$; then $e\in M'$. Suppose that $e\neq 1$. Then $\psi(1-e)>0$. We can write $\psi=\sum_{i\in I}\omega_{\zeta_i,\zeta_i}$ for certain $\zeta_i\in H$. Then for at least one $\zeta_i$, we have $((1-e)\zeta_i|\zeta_i)\neq 0$ so that $(1-e)\zeta_i\neq 0$. On the other hand, we have
    \[
        \forall y\in n_\psi:\norm{y\zeta_i}^2\leq \psi(y^*y)=\norm{\Lambda_\psi(y)}^2
    \]
    so that $\zeta_i\in D(H,\psi)$ and hence $e\zeta_i = \zeta_i$. This is a contradiction. Hence we must have $e = 1$ and $D(H,\psi)$ is dense in $H$.
\end{proof}
Let $\xi\in H$. By Proposition \ref{Chap3: Prop: 2}, (iii)*, $D(R^\psi(\xi)^*)$ is invariant under the action of $M'$. Hence the projection $p$ onto $\overline{D(R^\psi(\xi)^*)}$ is in $M$. Considered as an operator from $pH$ to $H_\psi$, $R^\psi(\xi)^*$ is closed and densely defined and hence $\abs{R^\psi(\xi)^*}^2$ exists as a positive self-adjoint operator on $pH$ which by Proposition \ref{Chap3: Prop: 2}, (iii)*, is affiliated with $pMp$. We denote by $\theta^\psi(\xi,\xi)$ the element of $\widehat{M}_+$ (the extended positive part of $M$) associated with the couple $(pH,\abs{R^\psi(\xi)^*}^2)$ as in [7, Example 1.2 and Lemma 1.4], i.e.
\begin{definition}\label{Chap3: Def: 5}
    For each $\xi\in H$, we denote by
    \[
        \theta^\psi(\xi,\xi)
    \]
    the element of $\widehat{M}_+$ characterized by
    \begin{equation}
        \forall \eta\in H: \inner{\omega_{\eta,\eta},\theta^\psi(\xi,\xi)}=\begin{cases}
            \norm{R^\psi(\xi)^*\eta}^2\quad \text{ if } \eta\in D(R^\psi(\xi)^*) \\
            \infty\quad \text{otherwise}
        \end{cases}.
    \end{equation}
\end{definition}
\begin{remark}
    If $\xi\in D(H,\psi)$, we simply have
    \begin{equation}
        \theta^\psi(\xi,\xi)=R^\psi(\xi)R^\psi(\xi)^*.
    \end{equation}
\end{remark}
\begin{proposition}\label{Chap3: Prop: 7}
    For all $\xi\in H$ and $x\in M$, we have
    \[
        \theta^\psi(x\xi,x\xi)=x\cdot \theta^\psi(\xi,\xi)\cdot x^*.
    \]
\end{proposition}
\begin{proof}
    For all $\eta\in H$, we have, using Proposition \ref{Chap3: Prop: 2}, (ii)*, and Definition \ref{Chap3: Def: 5}
    \[
        \begin{split}
            \inner{\omega_{\eta,\eta},\theta^\psi(x\xi,x\xi)}=&\inner{\omega_{x^*\eta,x^*\eta},\theta^\psi(\xi,\xi)}\\
            =&\inner{x^*\cdot\omega_{\eta,\eta}\cdot x,\theta^\psi(\xi,\xi)}\\
            =&\inner{\omega_{\eta,\eta},x\cdot\theta^\psi(\xi,\xi)\cdot x^*}
        \end{split}
    \]
    where the last equality simply follows from the definition of the operation $m\mapsto x\cdot m\cdot x^*$ in $\widehat{M}_+$.
\end{proof}
Recall that by [7, Proposition 1.10], every normal weight $\varphi$ has a unique extension, also denoted $\varphi$, to a normal weight on $\widehat{M}_+$.
\begin{definition}
    Let $\varphi$ be a normal weight on $M$. We define
    \[
        q_\varphi: H\to [0,\infty]
    \]
    by
    \begin{equation}
        q_\varphi(\xi)=\inner{\varphi,\theta^\psi(\xi,\xi)}, \xi\in H.
    \end{equation}
\end{definition}
\begin{proposition}\label{Chap3: Prop: 9}
    Let $\varphi$ be a normal weight on $M$. Then $q_\varphi$ is a l.s.c. quadratic form on $M$, i.e.
    \begin{enumerate}[(i)]
        \item $\forall \xi_1,\xi_2\in H:q_\varphi(\xi_1+\xi_2)+q_\varphi(\xi_1-\xi_2)=2q_\varphi(\xi_1)+2q_\varphi(\xi_2)$,
        \item $\forall \xi\in H \forall \lambda\in \mathbb{C}: q_\varphi(\lambda \xi)=\abs{\lambda}^2q_\varphi(\xi)$,
        \item $q_\varphi$ is lower semi-continuous.
    \end{enumerate}
\end{proposition}
\begin{proof}
    (ii) is immediate. For the proof of (i) and (iii), first suppose that $\varphi=\omega_{\eta,\eta}$ for some $\eta\in H$. Then
    \begin{equation}\label{Chap3: Eqn: 5}
        q_\varphi(\xi)=\inner{\omega_{\eta,\eta},\theta^\psi(\xi,\xi)}=\begin{cases}
            \norm{R^\psi(\xi)^*\eta}^2\quad \text{ if } \eta\in D(R^\psi(\xi)^*) \\
            \infty\quad \text{otherwise}
        \end{cases}.
    \end{equation}
    Let $\xi_1,\xi_2\in H$. We shall prove that
    \begin{equation}\label{Chap3: Eqn: 6}
        q_\varphi(\xi_1+\xi_2)+q_\varphi(\xi_1-\xi_2)\leq 2q_\varphi(\xi_1)+2q_\varphi(\xi_2).
    \end{equation}
    If either $\eta\in D(R^\psi(\xi_1)^*)$ or $\eta\in D(R^\psi(\xi_2)^*)$, the right hand side of \eqref{Chap3: Eqn: 6} is $+\infty$ and hence \eqref{Chap3: Eqn: 6} holds. Now suppose that $\eta\in D(R^\psi(\xi_1)^*)$ and $\eta\in D(R^\psi(\xi_2)^*)$. Then by Proposition \ref{Chap3: Prop: 2}, (i)*, also $\eta\in D(R^\psi(\xi_1+\xi_2)^*)$ and $\eta\in D(R^\psi(\xi_1-\xi_2)^*)$. Furthermore,
    \[
        \begin{split}
            &\norm{R^\psi(\xi_1+\xi_2)^*\eta}^2+\norm{R^\psi(\xi_1-\xi_2)^*\eta}^2\\
            =&\norm{R^\psi(\xi_1)^*\eta+R^\psi(\xi_2)^*\eta}^2+\norm{R^\psi(\xi_1)^*\eta-R^\psi(\xi_2)^*\eta}^2\\
            =&2\norm{R^\psi(\xi_1)^*\eta}^2+2\norm{R^\psi(\xi_2)^*\eta}^2.
        \end{split}
    \]
    Thus we have proved \eqref{Chap3: Eqn: 6} in all cases.\par
    By \eqref{Chap3: Eqn: 6} applied to $\xi_1+\xi_2$ and $\xi_1-\xi_2$ we get
    \[
        4(q_\varphi(\xi_1)+q_\varphi(\xi_2))=q_\varphi(2\xi_1)+q_\varphi(2\xi_2)\leq 2q_\varphi(\xi_1+\xi_2)+2q_\varphi(\xi_1-\xi_2).
    \]
    In all, we have shown (i).\par
    By \eqref{Chap3: Eqn: 5}, we have
    \[
        \begin{split}
            \inner{\omega_{\eta,\eta},\theta^\psi(\xi,\xi)}=&\sup\{\abs{(R^\psi(\xi)^*\eta|\zeta)}^2|\zeta\in D(R^\psi(\xi)),\norm{\zeta}\leq 1\}\\
            =&\sup\{\abs{(\eta|R^\psi(\xi)\Lambda_\psi(y))}^2|y\in n_\psi,\norm{\Lambda_\psi(y)}\leq 1\}\\
            =&\sup\{\abs{(\eta|y\xi)}^2|y\in n_\psi,\norm{\Lambda_\psi(y)}\leq 1\}
        \end{split}
    \]
    for all $\xi\in H$. Since each $\xi\mapsto \abs{(\eta|y\xi)}^2$ is continuous, this implies (iii).\par
    Now let $\varphi$ be an arbitrary normal weight. Then we can write
    \[
        \varphi=\sum_{i\in I}\omega_{\eta_i,\eta_i}
    \]
    and thus (cf. the proof of [7, Proposition 1.10])
    \[
        \forall \xi\in H:  q_\varphi(\xi)=\inner{\varphi,\theta^\psi(\xi,\xi)}=\sum_{i=I}\inner{\omega_{\eta_i,\eta_i},\theta^\psi(\xi,\xi)}.
    \]
    Now (i) and (iii) follow by the first part of the proof.
\end{proof}
\begin{remark}
    Let $\varphi$ be a normal weight on $M$. Write
    \begin{equation}
        \Dom(q_\varphi)=\{\xi\in H|q_\varphi(\xi)<\infty\}.
    \end{equation}
    Then for all $x\in n_\varphi$ and $\xi\in D(H,\psi)$, we have
    \begin{equation}
        x^*\xi\in \Dom(q_\varphi).
    \end{equation}
    Indeed,
    \[
        \begin{split}
            q_\varphi(x^*\xi)=&\inner{\varphi,\theta^\psi(x^*\xi,x^*\xi)}\\
            =&\inner{\varphi,x^*\cdot\theta^\psi(\xi,\xi)\cdot x}\\
            \leq &\norm{\theta^\psi(\xi,\xi)}\inner{\varphi,x^*x}<\infty.
        \end{split}
    \]
    In particular, if $\varphi$ is semifinite then $\Dom(q_\varphi)$ is dense in $H$ (since $n_\varphi^*$ is strongly dense in $M$).
\end{remark}
\begin{definition}\label{Chap3: Def: 11}
    For each normal weight $\varphi$ on $M$, we define the spatial derivative $\dv{\varphi}{\psi}$ as the unique element of $\widehat{B(H)}_+$ such that
    \begin{equation}
        \forall \xi\in H: \inner{\omega_{\xi,\xi},\dv{\varphi}{\psi}}=\inner{\varphi,\theta^\psi(\xi,\xi)}.
    \end{equation}
\end{definition}
The existence of $\dv{\varphi}{\psi}$ follows from Proposition \ref{Chap3: Prop: 9} and [7, proof of Lemma 1.4].
\begin{remark}
    If $\varphi$ is semifinite, $\dv{\varphi}{\psi}$ is simply a positive self-adjoint operator on $H$ (since in this case, $\{\xi\in H|\inner{\omega_{\xi,\xi},\dv{\varphi}{\psi}}<\infty\}=\Dom(q_\varphi)$ is dense in $H$). Note that
    \begin{equation}
        \forall \xi\in H:q_\varphi(\xi)=\begin{cases}
            \norm{\left( \dv{\varphi}{\psi} \right)^\frac{1}{2}\xi}^2\quad if \xi\in D\left( \left( \dv{\varphi}{\psi} \right)^\frac{1}{2} \right) \\
            \infty\quad \text{otherwise}
        \end{cases}.
    \end{equation}
\end{remark}
We shall see below (Proposition \ref{Chap3: Prop: 22} that the definition of $\dv{\varphi}{\psi}$ given here agrees with that given in [1]. (This is not quite obvious. Note that in [1, Lemma 6], the quadratic form $q$ is only defined on the subspace $D(H,\psi)$, and then extended by [1, Lemma 5] to the whole of $H$.)
\begin{lemma}\label{Chap3: Lemma: 13}
    Let $\varphi_1,\varphi_2,(\varphi_i)_{i\in I}$, and $\varphi$ be normal weights on $M$ and let $x\in M$. Then
    \begin{enumerate}[(i)]
        \item $\forall m\in \widehat{M}_+: \inner{\varphi_1+\varphi_2,m}=\inner{\varphi_1,m}+\inner{\varphi_2,m}$,
        \item $\forall m\in \widehat{M}_+: \inner{x\cdot\varphi\cdot x^*,m}=\inner{\varphi,x^*\cdot m\cdot x}$,
        \item if $\varphi_i\nearrow \varphi$, then $\forall m\in \widehat{M}_+:\inner{\varphi_i,m}\nearrow \inner{\varphi,m}$.
    \end{enumerate}
\end{lemma}
\begin{proof}
    (i) and (ii) are immediate consequences of [7, Proposition 1.10] (or its proof). As for (iii), we have by the proof of [7, Proposition 1.10], using the notation from there,
    \[
        \begin{split}
            \inner{\varphi_i,m}=&\sup_n\inner{\varphi_i,\int_0^n \lambda\dd e_\lambda}+\infty\cdot\varphi_i(p)\\
            \nearrow&\sup_n\inner{\varphi,\int_0^n\lambda\dd e_\lambda}+\infty\cdot\varphi_i(p)=\inner{\varphi,m}.
        \end{split}
    \]
\end{proof}
\begin{theorem}
    For all normal weights $\varphi_1$, $\varphi_2$, and $\varphi$ on $M$ and all $x\in M$ we have
    \begin{enumerate}[(a)]
        \item $\dv{(\varphi_1+\varphi_2)}{\psi}=\dv{\varphi_1}{\psi}+\dv{\varphi_2}{\psi}$,
        \item $\dv{(x\cdot\varphi\cdot x^*)}{\psi}=x\cdot\dv{\varphi}{\psi}\cdot x^*$.
    \end{enumerate}
\end{theorem}
\begin{remark}
    The sums and products occurring at the right hand side of (a) and (b) are to be understood in the sense of the operations in $\widehat{B(H)}_+$. In particular, if $\varphi_1$, $\varphi_2$, $\varphi_1+\varphi_2$ are semifinite, $\dv{\varphi_1}{\psi}+\dv{\varphi_2}{\psi}$ is the form sum of the positive self-adjoint operators $\dv{\varphi_1}{\psi}$ and $\dv{\varphi_2}{\psi}$. Similarly, if $x\cdot\varphi\cdot x^*$ is semifinite, $x\cdot\dv{\varphi}{\psi}\cdot x^*$ is the form product.
\end{remark}
\begin{remark}
    In [1], the sum property is simply stated without proof. It seems to be difficult to give a proof using only the methods of [1] (one only gets "$\geq$"). - The product property is stated (and proved) only for invertible $x\in M$.
\end{remark}
\begin{proof}[Proof of Theorem 14]
    Let $\xi\in H$. Then, using successively Definition \ref{Chap3: Def: 11}, Lemma \ref{Chap3: Lemma: 13}, Definition \ref{Chap3: Def: 11} again, and the definition of the sum in $\widehat{B(H)}_+$, we get
    \[
        \begin{split}
            \inner{\omega_{\xi,\xi},\dv{(\varphi_1+\varphi_2)}{\psi}}=&\inner{\varphi_1+\varphi_2,\theta^\psi(\xi,\xi)}\\
            =&\inner{\varphi_1,\theta^\psi(\xi,\xi)}+\inner{\varphi_2,\theta^\psi(\xi,\xi)}\\
            =&\inner{\omega_{\xi,\xi},\dv{\varphi_1}{\psi}}+\inner{\omega_{\xi,\xi},\dv{\varphi_2}{\psi}}\\
            =&\inner{\omega_{\xi,\xi},\dv{\varphi_1}{\psi}+\dv{\varphi_2}{\psi}}.
        \end{split}
    \]
    Similarly,
    \[
        \begin{split}
            \inner{\omega_{\xi,\xi},\dv{(x\cdot\varphi\cdot x^*)}{\psi}}=&\inner{x\cdot\varphi\cdot x^*,\theta^\psi(\xi,\xi)}=\inner{\varphi,x^*\cdot\theta^\psi(\xi,\xi)\cdot x}\\
            =&\inner{\varphi,\theta^\psi(x^*\xi,x^*\xi)}=\inner{\omega_{x^*\xi,x^*\xi},\dv{\varphi}{\psi}}\\
            =&\inner{x^*\cdot\omega_{\xi,\xi}\cdot x,\dv{\varphi}{\psi}}=\inner{\omega_{\xi,\xi},x\cdot\dv{\varphi_1}{\psi}\cdot x^*}
        \end{split}
    \]
    where we have used Lemma \ref{Chap3: Lemma: 13} and Proposition \ref{Chap3: Prop: 7}.
\end{proof}
\begin{theorem}
    Let $(\varphi_i)_{i\in I}$ and $\varphi$ be normal weights on $M$. Suppose that
    \[
        \varphi_i\nearrow \varphi.
    \]
    Then
    \[
        \dv{\varphi_i}{\psi}\nearrow\dv{\varphi}{\psi}.
    \]
\end{theorem}
\begin{remark}
    In particular, if $\varphi$ is semifinite, we have $\dv{\varphi_i}{\psi}\nearrow\dv{\varphi}{\psi}$ in the usual sense of positive self-adjoint operators.
\end{remark}
\begin{proof}[Proof of Theorem 17]
    For all $\xi\in H$, we have by Lemma \ref{Chap3: Lemma: 13}
    \[
        \begin{split}
            \inner{\omega_{\xi,\xi},\dv{\varphi_i}{\psi}}=&\inner{\varphi_i,\theta^\psi(\xi,\xi)}\\
            \nearrow&\inner{\varphi,\theta^\psi(\xi,\xi)}=\inner{\omega_{\xi,\xi},\dv{\varphi}{\psi}}.
        \end{split}
    \]
\end{proof}
\begin{lemma}\label{Chap3: Lemma: 19}
    Let $\varphi$ be a normal semifinite weight on $M$. Write $p=\supp\varphi$. Then for all $m\in \widehat{M}_+$, we have
    \[
        \inner{\varphi,m}=0\Leftrightarrow p\cdot m\cdot p=0.
    \]
\end{lemma}
\begin{proof}
    Let $m=\int_0^\infty \lambda\dd e_\lambda+\infty\cdot (1-r)$ be the spectral resolution of $m$. Put $x_n=\int_0^n \lambda\dd e_\lambda,n\in \mathbb{N}$. Then
    \[
        \begin{split}
            \inner{\varphi,m}=0\Leftrightarrow&\forall n\in \mathbb{N}:\inner{\varphi,x_n}=0\text{ and }\inner{\varphi,1-r}=0\\
            &\forall n\in \mathbb{N}:p\cdot x_n\cdot p=0\text{ and }p\cdot(1-r)\cdot p=0\\
            &\Leftrightarrow p\cdot m\cdot p=0.
        \end{split}
    \]
\end{proof}
\begin{theorem}
    Let $\varphi$ be a normal semifinite weight on $M$. Then
    \begin{equation}
        \supp\left( \dv{\varphi}{\psi} \right)=\supp(\varphi).
    \end{equation}
    In particular, $\dv{\varphi}{\psi}$ is injective if and only if $\varphi$ is faithful.
\end{theorem}
\begin{proof}
    Put $p=\supp\varphi\in M$. Now for all $\xi\in H$, we have, using Lemma \ref{Chap3: Lemma: 19} and Proposition \ref{Chap3: Prop: 7}:
    \[
        \begin{split}
            \xi\in\ker(\dv{\varphi}{\psi})\Leftrightarrow&\inner{\omega_{\xi,\xi},\dv{\varphi}{\psi}}=0\\
            \Leftrightarrow&\inner{\varphi,\theta^\psi(\xi,\xi)}=0\\
            \Leftrightarrow&p\cdot\theta^\psi(\xi,\xi)\cdot p=0\\
            \Leftrightarrow&\theta^\psi(p\xi,p\xi)=0\\
            \Leftrightarrow& p\xi=0\\
            \Leftrightarrow& \xi\in (1-p)H.
        \end{split}
    \]
    Since $ker\left( \dv{\varphi}{\psi} \right)=\supp\left( \dv{\varphi}{\psi} \right)^\perp$, the result follows.
\end{proof}
\begin{proposition}\label{Chap3: Prop: 22}

\end{proposition}
% \end{document}