% \documentclass[12pt]{report}
% \usepackage{geometry}
%  \geometry{
%  a4paper,
%  total={170mm,257mm},
%  left=20mm,
%  top=20mm,
%  }
% \usepackage[hidelinks]{hyperref}
% \usepackage{tikz-cd}
% \usepackage{amsthm}
% \usepackage{amsmath}
% \usepackage{amssymb}
% \usepackage{amsfonts}
% \usepackage{xcolor}
% \usepackage{physics}
% \usepackage{mathrsfs}
% \usepackage{bbm}

% \tikzcdset{every label/.append style = {font = \normalsize}}
% \newtheorem{theorem}{Theorem}[section]
% \newtheorem{proposition}[theorem]{Proposition}
% \newtheorem{corollary}[theorem]{Corollary}
% \newtheorem{definition}[theorem]{Definition}
% \theoremstyle{definition}
% \newtheorem{example}[theorem]{Example}
% \newcommand{\inner}[1]{\langle#1\rangle}
% \newcommand{\Sp}{\operatorname{Sp}}
% \begin{document}
% \tableofcontents
\chapter*{Introduction}
\addcontentsline{toc}{chapter}{Introduction} 
The main part of these notes (Chapter II) is devoted to a complete and detailed exposition of the theory of abstract $L^p$ spaces associated with von Neumann algebras. This theory was developed by U. Haagerup some seven years ago and outlined in a preprint (which now appears in \cite{9}). Unfortunately, in spite of his intentions, Haagerup has not yet had the time for writing down his theory in full. This is our motivation for writing these notes.\par
The proofs that we give are (close to) those that Haagerup originally had in mind and which he has told us at various occasions.\par
Essential for the construction of the $L^p$ spaces is the theory of measurable operators with respect to a trace on a von Neumann algebra (due to E. Nelson \cite{13} and inspired by \cite{15} and \cite{16}); we treat this in Chapter I. Other prerequisites are the basic facts on crossed products of a von Neumann algebra with a modular automorphism group and some results on operator valued weights and the extended positive part of a von Neumann algebra; we have not included this in the text but we give detailed references, especially to parts of (\cite{7} and \cite{8}, at the places where it is needed.\par
After the appearance of Haagerup's $L^p$ spaces, A. Connes proposed a definition of spatial $L^p$ spaces based on the notion of spatial derivatives \cite{1}. These spaces have been studied by H. Hilsum \cite{10}. We include a discussion of them and show how their main properties follow easily from the corresponding properties of Haagerup's spaces (thus our presentation is complementary to Hilsum's work \cite{10} where the objective is to develop the theory directly based on properties of spatial derivatives, avoiding as far as possible the dependence of Haagerup's construction). This is contained in Chapter IV.\par
Before this, we recall the main properties of spatial derivatives (Chapter III). We profit from this occasion to present a definition (due to U. Haagerup) of spatial derivatives that is slightly different from that given in \cite{1} and to show how certain properties (such as the sum property) of spatial derivatives are almost immediate consequences of this new definition.\par
The reader will notice that these notes do not contain a special chapter on the - now classic - theory of spaces with respect to a trace, due - in various formulations - to J. Dixmier \cite{3}  and R. A. Kunze \cite{12} (see also \cite{21} and \cite{13}). Although this important particular case has been motivating for the development of the more general theory, we do not directly need it in our preliminaries.For the sake of completeness, however, we give the definition of $L^p$ spaces with respect to a trace at the end of Chapter I, and in the following chapters, we point out how results concerning the trace case are related to the general results.\par
Another omission in these notes is the recent definition of $L^p$ spaces as complex interpolation spaces. For this, we simply refer to \cite{11} and \cite{20}.
% \end{document}