% \documentclass[12pt]{report}
% \usepackage{geometry}
%  \geometry{
%  a4paper,
%  total={170mm,257mm},
%  left=20mm,
%  top=20mm,
%  }
% \usepackage[hidelinks]{hyperref}
% \usepackage{tikz-cd}
% \usepackage{amsthm}
% \usepackage{amsmath}
% \usepackage{amssymb}
% \usepackage{amsfonts}
% \usepackage{xcolor}
% \usepackage{physics}
% \usepackage{mathrsfs}
% \usepackage{bbm}

% \tikzcdset{every label/.append style = {font = \normalsize}}
% \newtheorem{theorem}{Theorem}[section]
% \newtheorem{proposition}[theorem]{Proposition}
% \newtheorem{corollary}[theorem]{Corollary}
% \newtheorem{definition}[theorem]{Definition}
% \theoremstyle{definition}
% \newtheorem{example}[theorem]{Example}
% \newcommand{\inner}[1]{\langle#1\rangle}
% \newcommand{\Sp}{\operatorname{Sp}}
% \begin{document}
% \tableofcontents
\chapter{Spatial $L^p$ Spaces}
In this chapter, we describe the Connes/Hilsum construction of spatial $L^p$ spaces. \par
Let $M$ be a von Neumann algebra acting on a Hilbert space $H$ and let $\psi_0$ be a normal faithful semifinite weight on the commutant $M'$ of $M$.\par
The notation is as in Chapter II and III.
\begin{definition}
    For each positive self-adjoint $(-1)$-homogeneous operator $a$ we define the integral with respect to $\psi_0$ by
    \begin{equation}
        \int a\dd \psi_0=\varphi(1),
    \end{equation}
    where $\varphi$ is the (unique) normal semifinite weight on $M$ such that $a=\dv{\varphi}{\psi_0}$.
\end{definition}
\begin{notation}
    For each $p\in [1,\infty]$, we denote by
    \[
        \overline{M}_{-1/p}
    \]
    the set of closed densely defined $(-1/p)$-homogeneous operators on $H$.
\end{notation}
\begin{definition}
    Let $p\in[1,\infty[$. We put
    \begin{equation}\label{Chap4: Eqn: 2}
        L^p(\psi_0)=L^p(M,H,\psi_0)=\{a\in \overline{M}_{-1/p} |\int\abs{a}^p\dd \psi_0<\infty\}
    \end{equation}
    and
    \begin{equation}
        \norm{a}_p=\left( \int\abs{a}^p\dd \psi_0 \right)^\frac{1}{p},a\in L^p(\psi_0).
    \end{equation}
    For $p = \infty$, we put
    \begin{equation}
        L^\infty(\psi_0)=M
    \end{equation}
    and write $\norm{\cdot}_\infty$ for the usual operator norm on $M$.
\end{definition}
Note that when $a$ is $(-1/p)$-homogeneous, the operator $\abs{a}^p$ is $(-1)$-homogeneous so that the integral occurring at the right hand side of \eqref{Chap4: Eqn: 2} is defined.\par
The spaces $L^p(\psi_0)$ are called spatial $L^p$ spaces (as opposed to the abstract $L^p$ spaces of Haagerup).\par
We now follow the first part of [10] to describe the relationship between the $L^p(\psi_0)$ and Haagerup's $L^p(M)$.\par
Let $\varphi_0$ be a normal faithful semifinite weight on $M$. Put
\begin{equation}
    d_0=\dv{\varphi_0}{\psi_0}.
\end{equation}
Then
\begin{equation}
    \forall t\in \mathbb{R}\forall x\in M:\sigma_t^{\varphi_0}(x)=d_0^{it}xd_0^{-it}.
\end{equation}
We define a unitary operator $u_0$ on the Hilbert space $L^2(\mathbb{R},H)$ by
\begin{equation}
    (u_0\xi)(t)=d_0^{it}\xi(t),\xi\in L^2(\mathbb{R},H),t\in \mathbb{R}.
\end{equation}
Recall that the crossed product $N=R(M,\sigma^{\varphi_0})$ is generated by the elements $\pi(x),x\in M$, and $\lambda(s),s\in \mathbb{R}$, as described in the beginning of Chapter II. We shall describe the action of $u_0(\cdot)u_0^*$ on these generating elements.\par
By $\ell(s), s\in \mathbb{R}$, we denote the operator of translation by $s$ in $L^2(\mathbb{R})$:
\[
    (\ell(s)f)(t)=f(t-s),f\in L^2(\mathbb{R}),t\in \mathbb{R}.
\]\par
We identify $L^2(\mathbb{R},H)$ with $H\otimes L^2(\mathbb{R})$ (so that $v\otimes f,v\in H,f\in L^2(\mathbb{R})$, is identified with $\xi\in L^2(\mathbb{R},H)$ given by $\xi(t)=f(t)v,t\in \mathbb{R})$.
\begin{proposition}\label{Chap4: Prop: 3}
    1) For all $x\in M$, we have
    \[
        u_0\pi(x)u_0^*=x\otimes 1.
    \]

    2) For all $s\in \mathbb{R}$, we have
    \[
        u_0\lambda(s)u_0^*=d_0^{is}\otimes \ell(s).
    \]
\end{proposition}
\begin{proof}
    Let $\xi\in L^2(\mathbb{R},H)$. Then
    \[
        \begin{split}
            (u_0\pi(x)u_0^*\xi)(t)=&d_0^{it}\sigma_{-t}^{\varphi_0}(x)d_0^{-it}\xi(t)\\
            =&d_0^{it}d_0^{-it}xd_0^{it}d_0^{-it}\xi(t)\\
            =&x\xi(t),t\in \mathbb{R},
        \end{split}
    \]
    and
    \[
        \begin{split}
            (u_0\lambda(s)u_0^*\xi)(t)=&d_0^{it}(u_0^*\xi)(t-s)\\
            =&d_0^{it}d^{-i(t-s)}\xi(t-s)\\
            =&d_0^{is}\xi(t-s),t\in \mathbb{R}.
        \end{split}
    \]
    This proves the result since for $\xi=v\otimes f,v\in H,f\in L^2(\mathbb{R})$, we have
    \[
        ((x\otimes 1)(v\otimes f))(t)=(xv\otimes f)(t)=f(t)xv=xf(t)v=x\xi(t),t\in \mathbb{R},
    \]
    and
    \[
        \begin{split}
            ((d_0^{is}\otimes\ell(s))(v\otimes f))(t)=&(d_0^{is}v\otimes \ell(s)f)(t)\\
            =&(\ell(s)f)(t)d_0^{is}v\\
            =&f(t-s)d_0^{is}v\\
            =&d_0^{is}\xi(t-s),t\in \mathbb{R}.
        \end{split}
    \]
\end{proof}
We denote by $T$ the unique positive self-adjoint operator in $L^2(\mathbb{R})$ characterized by
\begin{equation}
    \forall s\in \mathbb{R}: T^{is}=\ell(s).
\end{equation}
For the definition and properties of tensor products of closed operators we refer to [17, Section 9.33].
\begin{proposition}
    For all normal semifinite weights $\varphi$ on $M$ we have
    \begin{equation}\label{Chap4: Eqn: 9}
        u_0h_\varphi u_0^*=\dv{\varphi}{\psi_0}\otimes T.
    \end{equation}
\end{proposition}
\begin{proof}
    First suppose that $\varphi$ is faithful . Then
    \[
        h_\varphi^{it}h_{\varphi_0}^{-it}=(D\tilde{\varphi}:D\tau)_t(D\tau:D\tilde{\varphi}_0)_t=(D\tilde{\varphi}:D\tilde{\varphi}_0)_t=\pi((D\varphi:D\varphi_0)_t)
    \]
    and
    \[
        (D\varphi:D\varphi_0)_t=\left( \dv{\varphi}{\psi_0} \right)^{it}\left( \dv{\varphi_0}{\psi_0} \right)^{-it}
    \]
    for all $t\in \mathbb{R}$, so that by Proposition \ref{Chap4: Prop: 3} and the fact that $h_{\varphi_0}^{it}=\lambda(t)$ for all $t\in \mathbb{R}$, we get
    \[
        \begin{split}
            u_0h_\varphi^{it}u_0^*=&(u_0h_\varphi^{it}h_{\varphi_0}^{-it}u_0^*)(u_0h_{\varphi_0}^{-it}u_0^*)\\
            =&\left( \left( \dv{\varphi}{\psi_0}^{it}\right) \left( \dv{\varphi_0}{\psi_0}^{-it}  \right)\otimes 1 \right)\left( \left( \dv{\varphi_0}{\psi_0} \right)^{it}\otimes \ell(t) \right)\\
            =&\left( \dv{\varphi}{\psi_0} \right)^{it}\otimes T^{it}
        \end{split}
    \]
    for all $t\in \mathbb{R}$, and \eqref{Chap4: Eqn: 9} follows.\par
    In the general case, choose a normal semifinite weight $\chi$ with $\supp \chi=1-p$ where $p = \supp \varphi$. Then $\varphi^+ \chi$ is a normal faithful semifinite weight and hence, by the first part of the proof,
    \[
        u_0h_{\varphi+\chi}u_0^*=\dv{(\varphi+\chi)}{\psi_0}\otimes T.
    \]
    Since $p=\supp \dv{\varphi}{\psi}$ and $\pi(p)=\supp h\varphi$, this implies that
    \[
        \begin{split}
            u_0h_\varphi u_0^*=&u_0(\pi(p)\cdot h_{\varphi+\chi}\cdot \pi(p))u_0^*\\
            =&u_0\pi(p)u_0^*\cdot u_0 h_{\varphi+\chi}u_0^*\cdot u_0\pi(p)u_0^*\\
            =&(p\otimes 1)\cdot \left( \dv{(\varphi+\chi)}{\psi_0}\otimes T \right)\cdot (p\otimes 1)\\
            =&\left( p\cdot\dv{(\varphi+\chi)}{\psi_0}\cdot p \right)\otimes T=\dv{\varphi}{\psi_0}\otimes T.
        \end{split}
    \]
\end{proof}
\begin{corollary}
    The mapping
    \[
        a\mapsto u_0^*(a\otimes T)u_0
    \]
    is a bijection of the set of positive self-adjoint ($-1$)-homogeneous operators $a$ on $H$ onto the set of positive self-adjoint operators $h$ affiliated with $R(M,\sigma^{\varphi_0})$ satisfying
    \begin{equation}
        \forall s\in \mathbb{R}:\theta_s h=e^{-s}h.
    \end{equation}
    Furthermore,
    \begin{equation}
        \int a\dd \psi_0=\tr(u_0^* (a\otimes T)u_0)
    \end{equation}
    for all such $a$.
\end{corollary}
\begin{proof}
    Since the mapping in question is nothing but $\dv{\varphi}{\psi_0}\mapsto h_\varphi$, it is a bijection by Proposition \ref{Chap2: Prop: 4} in Chapter II. By definition, we have $\int \dv{\varphi}{\psi_0}\dd \psi_0=\varphi(1)=\tr(h_\varphi)$.
\end{proof}
\begin{corollary}
    Let $p\in [1,\infty[$. Let $a$ be a closed densely defined operator on $H$. Then\par
    1) $a\in \overline{M}_{-1/p}$ if and only if
    \[
        u_0^*(a\otimes T^{1/p})u_0\eta R(M,\sigma^{\varphi_0}),
    \]\par
    2) $a\in L^p(\psi_0)$ if and only if
    \[
        u_0^*(a\otimes T^{1/p})u_0\in L^p(M).
    \]\par
    For all $a\in L^p(\psi_0)$, we have
    \[
        \norm{a}_p=\norm{u_0^*(a\otimes T^{1/p})u_0}_p.
    \]
\end{corollary}
\begin{corollary}
    Let $p\in [1,\infty[$. Then the mapping
    \begin{equation}
        a\mapsto u_0^*(a\otimes T^{1/p})u_0
    \end{equation}
    is a bijection of $\overline{M}_{-1/p}$ onto the set of closed densely defined operators $h$ affiliated with $R(M,\sigma^{\varphi_0})$ satisfying
    \begin{equation}
        \forall s\in \mathbb{R}:\theta_s h=e^{-s/p}h.
    \end{equation}
\end{corollary}
% \end{document}