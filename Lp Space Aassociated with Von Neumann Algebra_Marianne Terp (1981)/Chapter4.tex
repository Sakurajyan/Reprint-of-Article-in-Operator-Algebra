% \documentclass[12pt]{report}
% \usepackage{geometry}
%  \geometry{
%  a4paper,
%  total={170mm,257mm},
%  left=20mm,
%  top=20mm,
%  }
% \usepackage[hidelinks]{hyperref}
% \usepackage{tikz-cd}
% \usepackage{amsthm}
% \usepackage{amsmath}
% \usepackage{amssymb}
% \usepackage{amsfonts}
% \usepackage{xcolor}
% \usepackage{physics}
% \usepackage{mathrsfs}
% \usepackage{bbm}

% \tikzcdset{every label/.append style = {font = \normalsize}}
% \newtheorem{theorem}{Theorem}[section]
% \newtheorem{proposition}[theorem]{Proposition}
% \newtheorem{corollary}[theorem]{Corollary}
% \newtheorem{definition}[theorem]{Definition}
% \theoremstyle{definition}
% \newtheorem{example}[theorem]{Example}
% \newcommand{\inner}[1]{\langle#1\rangle}
% \newcommand{\Sp}{\operatorname{Sp}}
% \begin{document}
% \tableofcontents
\chapter{Spatial $L^p$ Spaces}
In this chapter, we describe the Connes/Hilsum construction of spatial $L^p$ spaces. \par
Let $M$ be a von Neumann algebra acting on a Hilbert space $H$ and let $\psi_0$ be a normal faithful semifinite weight on the commutant $M'$ of $M$.\par
The notation is as in Chapter II and III.
\begin{definition}
    For each positive self-adjoint $(-1)$-homogeneous operator $a$ we define the integral with respect to $\psi_0$ by
    \begin{equation}
        \int a\dd \psi_0=\varphi(1),
    \end{equation}
    where $\varphi$ is the (unique) normal semifinite weight on $M$ such that $a=\dv{\varphi}{\psi_0}$.
\end{definition}
\begin{notation}
    For each $p\in [1,\infty]$, we denote by
    \[
        \overline{M}_{-1/p}
    \]
    the set of closed densely defined $(-1/p)$-homogeneous operators on $H$.
\end{notation}
\begin{definition}
    Let $p\in[1,\infty[$. We put
    \begin{equation}\label{Chap4: Eqn: 2}
        L^p(\psi_0)=L^p(M,H,\psi_0)=\{a\in \overline{M}_{-1/p} |\int\abs{a}^p\dd \psi_0<\infty\}
    \end{equation}
    and
    \begin{equation}
        \norm{a}_p=\left( \int\abs{a}^p\dd \psi_0 \right)^\frac{1}{p},a\in L^p(\psi_0).
    \end{equation}
    For $p = \infty$, we put
    \begin{equation}
        L^\infty(\psi_0)=M
    \end{equation}
    and write $\norm{\cdot}_\infty$ for the usual operator norm on $M$.
\end{definition}
Note that when $a$ is $(-1/p)$-homogeneous, the operator $\abs{a}^p$ is $(-1)$-homogeneous so that the integral occurring at the right hand side of \eqref{Chap4: Eqn: 2} is defined.\par
The spaces $L^p(\psi_0)$ are called spatial $L^p$ spaces (as opposed to the abstract $L^p$ spaces of Haagerup).\par
We now follow the first part of [10] to describe the relationship between the $L^p(\psi_0)$ and Haagerup's $L^p(M)$.\par
Let $\varphi_0$ be a normal faithful semifinite weight on $M$. Put
\begin{equation}
    d_0=\dv{\varphi_0}{\psi_0}.
\end{equation}
Then
\begin{equation}
    \forall t\in \mathbb{R}\forall x\in M:\sigma_t^{\varphi_0}(x)=d_0^{it}xd_0^{-it}.
\end{equation}
We define a unitary operator $u_0$ on the Hilbert space $L^2(\mathbb{R},H)$ by
\begin{equation}\label{Chap4: Eqn: 7}
    (u_0\xi)(t)=d_0^{it}\xi(t),\xi\in L^2(\mathbb{R},H),t\in \mathbb{R}.
\end{equation}
Recall that the crossed product $N=R(M,\sigma^{\varphi_0})$ is generated by the elements $\pi(x),x\in M$, and $\lambda(s),s\in \mathbb{R}$, as described in the beginning of Chapter II. We shall describe the action of $u_0(\cdot)u_0^*$ on these generating elements.\par
By $\ell(s), s\in \mathbb{R}$, we denote the operator of translation by $s$ in $L^2(\mathbb{R})$:
\[
    (\ell(s)f)(t)=f(t-s),f\in L^2(\mathbb{R}),t\in \mathbb{R}.
\]\par
We identify $L^2(\mathbb{R},H)$ with $H\otimes L^2(\mathbb{R})$ (so that $v\otimes f,v\in H,f\in L^2(\mathbb{R})$, is identified with $\xi\in L^2(\mathbb{R},H)$ given by $\xi(t)=f(t)v,t\in \mathbb{R})$.
\begin{proposition}\label{Chap4: Prop: 3}
    1) For all $x\in M$, we have
    \[
        u_0\pi(x)u_0^*=x\otimes 1.
    \]

    2) For all $s\in \mathbb{R}$, we have
    \[
        u_0\lambda(s)u_0^*=d_0^{is}\otimes \ell(s).
    \]
\end{proposition}
\begin{proof}
    Let $\xi\in L^2(\mathbb{R},H)$. Then
    \[
        \begin{split}
            (u_0\pi(x)u_0^*\xi)(t)=&d_0^{it}\sigma_{-t}^{\varphi_0}(x)d_0^{-it}\xi(t)\\
            =&d_0^{it}d_0^{-it}xd_0^{it}d_0^{-it}\xi(t)\\
            =&x\xi(t),t\in \mathbb{R},
        \end{split}
    \]
    and
    \[
        \begin{split}
            (u_0\lambda(s)u_0^*\xi)(t)=&d_0^{it}(u_0^*\xi)(t-s)\\
            =&d_0^{it}d^{-i(t-s)}\xi(t-s)\\
            =&d_0^{is}\xi(t-s),t\in \mathbb{R}.
        \end{split}
    \]
    This proves the result since for $\xi=v\otimes f,v\in H,f\in L^2(\mathbb{R})$, we have
    \[
        ((x\otimes 1)(v\otimes f))(t)=(xv\otimes f)(t)=f(t)xv=xf(t)v=x\xi(t),t\in \mathbb{R},
    \]
    and
    \[
        \begin{split}
            ((d_0^{is}\otimes\ell(s))(v\otimes f))(t)=&(d_0^{is}v\otimes \ell(s)f)(t)\\
            =&(\ell(s)f)(t)d_0^{is}v\\
            =&f(t-s)d_0^{is}v\\
            =&d_0^{is}\xi(t-s),t\in \mathbb{R}.
        \end{split}
    \]
\end{proof}
We denote by $T$ the unique positive self-adjoint operator in $L^2(\mathbb{R})$ characterized by
\begin{equation}
    \forall s\in \mathbb{R}: T^{is}=\ell(s).
\end{equation}
For the definition and properties of tensor products of closed operators we refer to [17, Section 9.33].
\begin{proposition}\label{Chap4: Prop: 4}
    For all normal semifinite weights $\varphi$ on $M$ we have
    \begin{equation}\label{Chap4: Eqn: 9}
        u_0h_\varphi u_0^*=\dv{\varphi}{\psi_0}\otimes T.
    \end{equation}
\end{proposition}
\begin{proof}
    First suppose that $\varphi$ is faithful . Then
    \[
        h_\varphi^{it}h_{\varphi_0}^{-it}=(D\tilde{\varphi}:D\tau)_t(D\tau:D\tilde{\varphi}_0)_t=(D\tilde{\varphi}:D\tilde{\varphi}_0)_t=\pi((D\varphi:D\varphi_0)_t)
    \]
    and
    \[
        (D\varphi:D\varphi_0)_t=\left( \dv{\varphi}{\psi_0} \right)^{it}\left( \dv{\varphi_0}{\psi_0} \right)^{-it}
    \]
    for all $t\in \mathbb{R}$, so that by Proposition \ref{Chap4: Prop: 3} and the fact that $h_{\varphi_0}^{it}=\lambda(t)$ for all $t\in \mathbb{R}$, we get
    \[
        \begin{split}
            u_0h_\varphi^{it}u_0^*=&(u_0h_\varphi^{it}h_{\varphi_0}^{-it}u_0^*)(u_0h_{\varphi_0}^{-it}u_0^*)\\
            =&\left( \left( \dv{\varphi}{\psi_0}^{it}\right) \left( \dv{\varphi_0}{\psi_0}^{-it}  \right)\otimes 1 \right)\left( \left( \dv{\varphi_0}{\psi_0} \right)^{it}\otimes \ell(t) \right)\\
            =&\left( \dv{\varphi}{\psi_0} \right)^{it}\otimes T^{it}
        \end{split}
    \]
    for all $t\in \mathbb{R}$, and \eqref{Chap4: Eqn: 9} follows.\par
    In the general case, choose a normal semifinite weight $\chi$ with $\supp \chi=1-p$ where $p = \supp \varphi$. Then $\varphi^+ \chi$ is a normal faithful semifinite weight and hence, by the first part of the proof,
    \[
        u_0h_{\varphi+\chi}u_0^*=\dv{(\varphi+\chi)}{\psi_0}\otimes T.
    \]
    Since $p=\supp \dv{\varphi}{\psi}$ and $\pi(p)=\supp h\varphi$, this implies that
    \[
        \begin{split}
            u_0h_\varphi u_0^*=&u_0(\pi(p)\cdot h_{\varphi+\chi}\cdot \pi(p))u_0^*\\
            =&u_0\pi(p)u_0^*\cdot u_0 h_{\varphi+\chi}u_0^*\cdot u_0\pi(p)u_0^*\\
            =&(p\otimes 1)\cdot \left( \dv{(\varphi+\chi)}{\psi_0}\otimes T \right)\cdot (p\otimes 1)\\
            =&\left( p\cdot\dv{(\varphi+\chi)}{\psi_0}\cdot p \right)\otimes T=\dv{\varphi}{\psi_0}\otimes T.
        \end{split}
    \]
\end{proof}
\begin{corollary}\label{Chap4: Coro: 5}
    The mapping
    \[
        a\mapsto u_0^*(a\otimes T)u_0
    \]
    is a bijection of the set of positive self-adjoint ($-1$)-homogeneous operators $a$ on $H$ onto the set of positive self-adjoint operators $h$ affiliated with $R(M,\sigma^{\varphi_0})$ satisfying
    \begin{equation}
        \forall s\in \mathbb{R}:\theta_s h=e^{-s}h.
    \end{equation}
    Furthermore,
    \begin{equation}
        \int a\dd \psi_0=\tr(u_0^* (a\otimes T)u_0)
    \end{equation}
    for all such $a$.
\end{corollary}
\begin{proof}
    Since the mapping in question is nothing but $\dv{\varphi}{\psi_0}\mapsto h_\varphi$, it is a bijection by Proposition \ref{Chap2: Prop: 4} in Chapter II. By definition, we have $\int \dv{\varphi}{\psi_0}\dd \psi_0=\varphi(1)=\tr(h_\varphi)$.
\end{proof}
\begin{corollary}\label{Chap4: Coro: 6}
    Let $p\in [1,\infty[$. Let $a$ be a closed densely defined operator on $H$. Then\par
    1) $a\in \overline{M}_{-1/p}$ if and only if
    \[
        u_0^*(a\otimes T^{1/p})u_0\eta R(M,\sigma^{\varphi_0}),
    \]\par
    2) $a\in L^p(\psi_0)$ if and only if
    \[
        u_0^*(a\otimes T^{1/p})u_0\in L^p(M).
    \]\par
    For all $a\in L^p(\psi_0)$, we have
    \[
        \norm{a}_p=\norm{u_0^*(a\otimes T^{1/p})u_0}_p.
    \]
\end{corollary}
\begin{corollary}\label{Chap4: Coro: 7}
    Let $p\in [1,\infty[$. Then the mapping
    \begin{equation}
        a\mapsto u_0^*(a\otimes T^{1/p})u_0
    \end{equation}
    is a bijection of $\overline{M}_{-1/p}$ onto the set of closed densely defined operators $h$ affiliated with $R(M,\sigma^{\varphi_0})$ satisfying
    \begin{equation}
        \forall s\in \mathbb{R}:\theta_s h=e^{-s/p}h.
    \end{equation}
\end{corollary}
\begin{proof}[Proof of Corollary \ref{Chap4: Coro: 6} and \ref{Chap4: Coro: 7}]
    Let $a$ be a closed densely defined operator on $H$ with polar decomposition $a=u\abs{a}$. Then
    \[
        h=u_0^*(u\otimes 1)u_0(u_0^*(\abs{a}\otimes T)u_0)^{1/p}
    \]
    is the polar decomposition of $h = u_0^*(a\otimes T^{1/p})u_0$. Corollary \ref{Chap4: Coro: 6}, 1), and Corollary \ref{Chap4: Coro: 7} now follow from Corollary \ref{Chap4: Coro: 5} and Proposition \ref{Chap4: Prop: 3}, 1) (and the fact that $a\mapsto a\otimes T^{1/p}$ is injective). The rest of Corollary \ref{Chap4: Coro: 6} follows from the equation $\int\abs{a}^p \dd \psi_0=\tr(\abs{u_0^*(\abs{a}\otimes T^{1/p})u_0}^p)$.
\end{proof}
\begin{proposition}\label{Chap4: Prop: 8}
    Let $p\in [1,\infty]$. Then for all $a\in L^p(\psi_0)$, we have $a^*\in L^p(\psi_0)$ and
    \[
        \norm{a^*}_p=\norm{a}_p.
    \]
\end{proposition}
\begin{proof}
    Let $a\in L^p(\psi_0)$. Then $a\otimes T^{1/p}\in u_0L^p(M)u_0^*$. Hence also $a^*\otimes T^{1/p}=(a\otimes T^{1/p})^*\in u_0 L^p(M)u_0^*$. Thus $a^*\in L^p(\psi_0)$ by Corollary \ref{Chap4: Coro: 6} and $\norm{a^*}_p = \norm{u_0^*(a^*\otimes T^{1/p})u_0}_p=\norm{u_0^*(a\otimes T^{1/p})u_0}_p=\norm{a}_p$.
\end{proof}
If we identify $L^2(\mathbb{R})$ with $L^2(\mathbb{R})$ via Fourier transformation, $T$ is simply the multiplication operator in $L^2(\mathbb{R})$ given by multiplication by $t\mapsto e^t$, and similarly, for each $p\in [1,\infty[$, $T^{1/p}$ is simply multiplication by $t\mapsto e^{t/p}$. This observation will permit us to obtain information about operators $a$ on $H$ from information about the tensor products $a\otimes T^{1/p}$. First we have:
\begin{lemma}\label{Chap4: Lemma: 9}
    Let $a$ be a closed densely defined operator on $H$ and $f$ a Borel function on $\mathbb{R}$, and denote by $m_f$ the corresponding multiplication operator on $L^2(\mathbb{R})$. Write
    \[
        \begin{split}
            D = \{\xi\in L^2(\mathbb{R},H)|&\xi(t)\in D(a) \text{{\color{red} for  a.a.}} t\in \mathbb{R}\\
            &\text{ and }\int \norm{f(t)a\xi(t)}^2\dd t <\infty\}.
        \end{split}
    \]
    Then $D(a\otimes m_f)=D$ and
    \[
        ((a\otimes m_f)\xi)(t)=f(t)a\xi(t),\xi\in D,t\in \mathbb{R}.
    \]
\end{lemma}
\begin{proof}
    Denote by $m(a,f)$ the operator in $L^2(\mathbb{R},H)$ given by
    \[
        D(m(a,f)) = D
    \]
    and
    \[
        (m(a,f)\xi)(t)=f(t)a\xi(t), \xi\in D, t\in \mathbb{R}.
    \]
    Then $m(a,f)$ is a closed operator and
    \[
        m(a^*,\bar{f})\subset m(a,f)^*
    \]
    (in fact, equality holds). Now evidently
    \[
        a\odot m_f\subset m(a,f),
    \]
    where $a\odot m_f$ denotes the algebraic tensor product of $a$ and $m_f$, and hence
    \[
        a\otimes m_f=[a\odot m_f]\subset m(a,f).
    \]
    Applying this to $a^*$ and $\bar{f}$, we get
    \[
        a^*\otimes m_{\bar{f}}\subset m(a^*,\bar{f}).
    \]
    Combining this, and using that $(A\otimes B)^*=A^*\otimes B^*$, we find that
    \[
        m(a,f)\subset m(a^*,\bar{f})^*\subset (a^*\otimes m_{\bar{f}})^*=a\otimes m_f.
    \]
    In all, we have shown that $a\otimes m_f=m(a,f)$.
\end{proof}
\begin{lemma}\label{Chap4: Lemma: 10}
    Let $p\in [1,\infty]$ and $a,b\in L^p(\psi_0)$.
    Then $a+b$ is densely defined and preclosed, and
    \[
        [a+b]\in L^p(\psi_0).
    \]
\end{lemma}
\begin{proof}
    1) Denote by $e$ the projection onto $\overline{D(a)\cap D(b)}$. Then
    \[
        \begin{split}
            &(e\otimes 1)L^2(\mathbb{R},H)\\
            =&\{\xi\in L^2(\mathbb{R},H)|\xi(t)=e\xi(t) \text{ for a.a. } t\in \mathbb{R}\}\\
            \supset &\{\xi\in L^2(\mathbb{R},H)|\xi(t)\in D(a)\cap D(b)\text{ for a.a. } t\in \mathbb{R}\}
        \end{split}
    \]
    By Lemma \ref{Chap4: Lemma: 9}, this set contains
    \[
        D(a\otimes T^{1/p})\cap D(b\otimes T^{1/p}).
    \]
    Now since $a\otimes T^{1/p},b\otimes T^{1/p}\in u_0L^p(\psi)u_0^*$, their sum is densely defined. Hence $D(a\otimes T^{1/p})\cap D(b\otimes T^{1/p})$ is dense in $L^2(\mathbb{R},H)$. It follows that $e = 1$. Hence $D(a+b) = D(a)\cap D(b)$ is dense in $H$.\par
    2) Now let us show that $a+b$ is preclosed. By Proposition \ref{Chap4: Prop: 8}, $a^*$ and $b^*$ are in $L^p(\psi_0)$ and hence by the first part of proof, $a^*+ b^*$ is densely defined. Since $a+b \subset (a^*+b^*)^*$, $a+b$ is preclosed.\par
    3) Finally, let us show that
    \begin{equation}\label{Chap4: Eqn: 14}
        [a+b]\otimes T^{1/p}=[(a\otimes T^{1/p})+(b\otimes T^{1/p})].
    \end{equation}
    First, by the characterization of $a\otimes T^{1/p}$ given in Lemma \ref{Chap4: Lemma: 9} we obviously have
    \[
        (a\otimes T^{1/p})+(b\otimes T^{1/p})\subset [a+b]\otimes T^{1/p},
    \]
    whence
    \[
        [(a\otimes T^{1/p})+(b\otimes T^{1/p})]\subset [a+b]\otimes T^{1/p}.
    \]
    On the other hand, again by that characterization,
    \[
        [a+b]\otimes T^{1/p}\subset ((a^*\otimes T^{1/p})+(b^*\otimes T^{1/p}))^*,
    \]
    and finally
    \[
        ((a^*\otimes T^{1/p})+(b^*\otimes T^{1/p}))^*=[(a\otimes T^{1/p})+(b\otimes T^{1/p})]
    \]
    since * is an involution in $L^p(M)$ (and hence respects the strong sum). In all, we have proved \eqref{Chap4: Eqn: 14}. Now the right hand side of \eqref{Chap4: Eqn: 14} is in $u_0L^p(M)u_0^*$. Hence by Corollary \ref{Chap4: Coro: 6}, $[a+b]\in L^p(\psi_0)$.
\end{proof}
\begin{lemma}
    Let $p,p_1,p_2\in [1,\infty]$ such that $1/p = 1/p_1 + 1/p_2$. Let $a\in L^{p_1}(\psi_0)$ and $b \in L^{p_2}(\psi_0)$. Then $ab$ is densely defined and preclosed and
    \[
        [ab]\in L^p(\psi_0).
    \]
\end{lemma}
\begin{proof}
    1) Denote by $e$ the projection onto $D(ab)$. Then, using Lemma \ref{Chap4: Lemma: 9}, we have
    \[
        \begin{split}
            &D((a\otimes T^{1/p})(b\otimes T^{1/p}))\\
            \subset& \{\xi\in D(b\otimes T^{1/p})|b\xi(t)\in D(a) \text{ for a.a. }t\in \mathbb{R}\}\\
            \subset& \{\xi\in L^2(\mathbb{R},H)|\xi(t)\in D(b) \text{ for a.a. }t\in \mathbb{R}\\
            &\quad \text{ and } b\xi(t)\in D(a) \text{ for a.a. }t\in \mathbb{R}\}\\
            \subset& \{\xi\in L^2(\mathbb{R},H)|\xi(t)\in D(ab) \text{ for a.a. }t\in \mathbb{R}\}\\
            \subset& \{\xi\in L^2(\mathbb{R},H)|\xi(t)=e\xi(t) \text{ for a.a. }t\in \mathbb{R}\}\\
            =&(e\otimes 1)L^2(\mathbb{R},H).
        \end{split}
    \]

    Hence $e = 1$ and $ab$ is densely defined. \par
    2) By 1) applied to $b^*$ and $a^*$, $b^*a^*$ is densely defined. Since $ab\subset (b^*a^*)^*$, $ab$ is preclosed.\par
    3) Finally let us show that
    \[
        [ab]\otimes T^{1/p}=[(a\otimes T^{1/p})(b\otimes T^{1/p})].
    \]
    First, by Lemma \ref{Chap4: Lemma: 9},
    \[
        (a\otimes T^{1/p})(b\otimes T^{1/p})\subset [ab]\otimes T^{1/p},
    \]
    whence
    \[
        [(a\otimes T^{1/p})(b\otimes T^{1/p})] \subset [ab]\otimes T^{1/p}.
    \]
    On the other hand,
    \[
        [ab]\otimes T^{1/p}\subset ((b^*\otimes T^{1/p})(a^*\otimes T^{1/p}))^*=[(a\otimes T^{1/p})(b\otimes T^{1/p})].
    \]
    The result follows as in the proof of Lemma \ref{Chap4: Lemma: 10}.
\end{proof}
Now we are ready to transform the results on the spaces $L^p(M)$ obtained in Chapter II into results on the $L^p(\psi_0)$ (for an alternative, see [10]). \par
From Corollary \ref{Chap4: Coro: 7}, Corollary \ref{Chap4: Coro: 6}, 2), and Lemma \ref{Chap4: Lemma: 10} we now get:
\begin{theorem}\label{Chap4: Thm: 12}
    Let $p\in [1,\infty]$. Then $(L^p(\psi_0),\norm{\cdot}_p)$ is a Banach space with respect to strong sum. \par
    The mapping $a\to u_0^*(a\otimes T^{1/p})u_0$ is an isometric isomorphism of $L^p(\psi_0)$ onto $L^p(M)$.
\end{theorem}
\begin{notation}
    From now on, the strong product $[ab]$ of operators $a$ and $b$ will be denoted $a\cdot b$.
\end{notation}
\begin{proposition}
    Let $p,q\in [1,\infty]$ with $1/p + 1/q=1$. Then for all $a\in L^p(\psi_0)$ and $b\in L^q(\psi_0)$ we have
    \[
        \int a\cdot b\dd \psi_0= \int b\cdot a\dd \psi_0.
    \]
\end{proposition}
\begin{proof}
    We have $a\cdot b\in L^1(\psi_0)$ and $b\cdot a\in L^1(\psi_0)$, and
    \[
        \begin{split}
            \int a\cdot b\dd \psi_0=&\tr(u_0^*((a\otimes T^{1/p})\cdot (b\otimes T^{1/p}))u_0)\\
            =&\tr(u_0^*(a\otimes T^{1/p})u_0\cdot u_0^*(b\otimes T^{1/p})u_0)\\
            =&\tr(u_0^*(b\otimes T^{1/p})u_0\cdot u_0^*(a\otimes T^{1/p})u_0)\\
            =&\tr(u_0^*((b\otimes T^{1/p})\cdot (a\otimes T^{1/p}))u_0)\\
            =&\int b\cdot a\dd \psi_0.
        \end{split}
    \]
\end{proof}
The following results are now immediate corollaries of the corresponding results in Chapter II.
\begin{proposition}[H\"older's inequality]
    Let $p,q\in [1,\infty]$ with $1/p + 1/q = 1$. Then for all $a\in L^p(\psi_0)$ and $b\in L^q(\psi_0)$ we have
    \[
        \norm{a\cdot b}_1\leq \norm{a}_p\norm{b}_q.
    \]
\end{proposition}
\begin{theorem}
    Let $p\in [1,\infty[$ and define $q$ by $1/q=1-1/p$.\par
    1) For each $b\in L^q(\psi_0)$, the mapping $\varphi_b$ defined by
    \[
        \varphi_b(a)=\int a\cdot b\dd \psi_0,a\in L^p(\psi_0),
    \]
    is a bounded linear functional on $L^q(\psi_0)$.\par
    2) For all $b\in L^q(\psi_0)$ we have
    \[
        \norm{b}_q=\sup\{\abs{\int a\cdot b\dd \psi_0}|a\in L^p(\psi_0),\norm{a}_p\leq 1\}.
    \]\par
    3) The mapping
    \[
        b\mapsto \varphi_b
    \]
    is an isometric isomorphism of $L^q(\psi_0)$ onto the dual Banach space $L^p(\psi_0)$.\par
\end{theorem}
\begin{proposition}
    $L^2(\psi_0)$ is a Hilbert space with the inner product
    \[
        (a,b)\mapsto \int b^*\cdot a\dd \psi_0.
    \]
    We define left and right actions of $M$ on $L^2(\psi_0)$ by
    \[
        \begin{split}
            \lambda(x)a=x\cdot a,a\in L^2(\psi_0),\\
            \rho(x)a=a\cdot x,a\in L^2(\psi_0),
        \end{split}
    \]
    for all $x\in M$ (as usual, "$\cdot$" means "strong product").
\end{proposition}
\begin{proposition}
    The quadruple $(\lambda,L^2(\psi_0),*,L^2(\psi_0)_+)$ is a standard form of $M$ in the sense of [4].
\end{proposition}
\section{$L^p$ spaces with respect to a trace}
Suppose that $\tau$ is a normal faithful semifinite trace on $M$. Denote by $\tau'$ the trace on $M'$ associated with $\tau$ via $\dv{\tau}{\tau'}= 1$. Now for each $p\in [1,\infty]$, the ($-1/p$)-homogeneous operators with respect to $\tau'$ are precisely the operators affiliated with $M$. Let $a$ be a positive self-adjoint operator affiliated with $M$. Then
\begin{equation}
    \tau(a)=\int a\dd \tau',
\end{equation}
since $\tau(a)=\tau(a\cdot)(1)$ and (by Chapter III, Corollary \ref{Chap3: Coro: 32})
\begin{equation}
    \dv{\tau(a\cdot)}{\tau'}=a.
\end{equation}
It follows that for all $p\in[1,\infty]$, we have
\begin{equation}
    L^p(\tau')=L^p(M,\tau),
\end{equation}
where $L^p(\tau')$ is a spatial $L^p$ space as discussed in this chapter and $L^p(M,\tau)$ is as defined at the end of Chapter I. Hence $L^p$ spaces as defined in this chapter are generalizations of the well-known $L^p$ spaces with respect to a trace. On the other hand, all the results on $L^p$ spaces that we have proved in particular apply to $L^p$ spaces with respect to a trace, so that we have reproved the well-known properties of such spaces.
\section{Change of weight}
Let $\psi_0$ and $\psi_1$ be two normal faithful semifinite weights on $M'$. Then by Theorem \ref{Chap4: Thm: 12}, there exists an isometric isomorphism
\begin{equation}
    \Phi:L^p(\psi_0)\to L^p(\psi_1)
\end{equation}
characterized by
\begin{equation}
    \forall a\in L^p(\psi_0):u_1^*(\Phi(a)\otimes T^{1/p})u_1=u_0^*(a\otimes T^{1/p})u_0,
\end{equation}
where $u_1$ is the unitary on $L^2(\mathbb{R},H)$ constructed from $d_1=\dv{\varphi_0}{\psi_1}$ in analogy with \eqref{Chap4: Eqn: 7}.\par
For positive injective $a\in L^p(\psi_0)$, we have $\Phi(a)=b$, where $b$ is the positive self-adjoint operator on $H$ characterized by
\begin{equation}
    \forall t\in \mathbb{R}: b^{p~it} = d_1^{it}d_0^{-it} a^{p~it}.
\end{equation}
Indeed , if $\varphi$ is the normal faithful semifinite weight given by $a^p=\dv{\varphi}{\psi_0}$, then for all $t\in \mathbb{R}$ we have
\[
    \left( \dv{\varphi}{\psi_1} \right)^{it}=(D\psi_1:D\psi_0)_{-t}\left( \dv{\varphi}{\psi_0} \right)^{it}=\left( \dv{\varphi_0}{\psi_1} \right)^{it}\left( \dv{\varphi_0}{\psi_0} \right)^{-it}a^{p~it}=d_1^{it}d_0^{-it}a^{p~it}
\]
and
\[
    \begin{split}
        u_1^*\left( \left( \dv{\varphi}{\psi_1} \right)^{1/p}\otimes T^{1/p} \right)u_1=&\left( u_1^*\left( \dv{\varphi}{\psi_1}\otimes T \right)u_1 \right)^{1/p}\\
        =&h_\varphi^{1/p}\\
        =&\left( u_0^*\left( \dv{\varphi}{\psi_0}\otimes T \right) u_0 \right)^{1/p}\\
        =&u_0^*(a\otimes T^{1/p})u_0.
    \end{split}
\]
\section{Note}
The operators $h_\varphi$ from Chapter II are themselves spatial derivatives of $\varphi$ with respect to a certain weight $\chi_0$ on the commutant $u_0^*(M'\otimes B(L^2(\mathbb{R})))u_0$ of $\pi(M)$ (where $\pi(M)\subset R(M,\sigma^{\varphi_0})$ is acting on $L^2(\mathbb{R},H)$). Indeed, by Chapter III, Theorem \ref{Chap3: Thm: 29}, there exists a unique normal faithful semifinite weight $\chi_0$ on this commutant such that
\begin{equation}
    \forall t\in \mathbb{R}:\left( \dv{\varphi_0}{\chi_0} \right)^{it}=\lambda(t)=h_{\varphi_0}^{it}.
\end{equation}
It follows by Chapter III, Theorem \ref{Chap3: Thm: 25}, that
\[
    \forall t\in \mathbb{R}:\left( \dv{\varphi}{\chi_0} \right)^{it}=h_\varphi^{it}
\]
for all normal faithful semifinite weights $\varphi$ on $M$, and hence
\begin{equation}
    h_\varphi=\dv{\varphi}{\chi_0}
\end{equation}
for all such $\varphi$. By the usual methods (cf. the proofs of Chapter II, Lemma \ref{Chap2: lemma: 1}, and this chapter, Proposition \ref{Chap4: Prop: 4}), this also holds for all normal semifinite not necessarily faithful weights.\par
One can show that if $(M,H,J,P)$ is a standard form of $M$, then
\[
    \chi_0=u_0^*\cdot(\varphi_0(J\cdot J)\otimes \tr(T^{-1}\cdot))\cdot u_0,
\]
where $\tr$ is the usual trace on $B(L^2(\mathbb{R}))$.
% \end{document}