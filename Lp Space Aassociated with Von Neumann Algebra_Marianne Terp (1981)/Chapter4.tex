% \documentclass[12pt]{report}
% \usepackage{geometry}
%  \geometry{
%  a4paper,
%  total={170mm,257mm},
%  left=20mm,
%  top=20mm,
%  }
% \usepackage[hidelinks]{hyperref}
% \usepackage{tikz-cd}
% \usepackage{amsthm}
% \usepackage{amsmath}
% \usepackage{amssymb}
% \usepackage{amsfonts}
% \usepackage{xcolor}
% \usepackage{physics}
% \usepackage{mathrsfs}
% \usepackage{bbm}

% \tikzcdset{every label/.append style = {font = \normalsize}}
% \newtheorem{theorem}{Theorem}[section]
% \newtheorem{proposition}[theorem]{Proposition}
% \newtheorem{corollary}[theorem]{Corollary}
% \newtheorem{definition}[theorem]{Definition}
% \theoremstyle{definition}
% \newtheorem{example}[theorem]{Example}
% \newcommand{\inner}[1]{\langle#1\rangle}
% \newcommand{\Sp}{\operatorname{Sp}}
% \begin{document}
% \tableofcontents
\chapter{Spatial $L^p$ Spaces}
In this chapter, we describe the Connes/Hilsum construction of spatial $L^p$ spaces. \par
Let $M$ be a von Neumann algebra acting on a Hilbert space $H$ and let $\psi_0$ be a normal faithful semifinite weight on the commutant $M'$ of $M$.\par
The notation is as in Chapter II and III.
\begin{definition}
    For each positive self-adjoint $(-1)$-homogeneous operator $a$ we define the integral with respect to $\psi_0$ by
    \begin{equation}
        \int a\dd \psi_0=\varphi(1),
    \end{equation}
    where $\varphi$ is the (unique) normal semifinite weight on $M$ such that $a=\dv{\varphi}{\psi_0}$.
\end{definition}
\begin{notation}
    For each $p\in [1,\infty]$, we denote by
    \[
        \overline{M}_{-1/p}
    \]
    the set of closed densely defined $(-1/p)$-homogeneous operators on $H$.
\end{notation}
\begin{definition}
    Let $p\in[1,\infty[$. We put
    \begin{equation}\label{Chap4: Eqn: 2}
        L^p(\psi_0)=L^p(M,H,\psi_0)=\{a\in \overline{M}_{-1/p} |\int\abs{a}^p\dd \psi_0<\infty\}
    \end{equation}
    and
    \begin{equation}
        \norm{a}_p=\left( \int\abs{a}^p\dd \psi_0 \right)^\frac{1}{p},a\in L^p(\psi_0).
    \end{equation}
    For $p = \infty$, we put
    \begin{equation}
        L^\infty(\psi_0)=M
    \end{equation}
    and write $\norm{\cdot}_\infty$ for the usual operator norm on $M$.
\end{definition}
Note that when $a$ is $(-1/p)$-homogeneous, the operator $\abs{a}^p$ is $(-1)$-homogeneous so that the integral occurring at the right hand side of \eqref{Chap4: Eqn: 2} is defined.\par
The spaces $L^p(\psi_0)$ are called spatial $L^p$ spaces (as opposed to the abstract $L^p$ spaces of Haagerup).\par
We now follow the first part of [10] to describe the relationship between the $L^p(\psi_0)$ and Haagerup's $L^p(M)$.\par
Let $\varphi_0$ be a normal faithful semifinite weight on $M$. Put
\begin{equation}
    d_0=\dv{\varphi_0}{\psi_0}.
\end{equation}
Then
\begin{equation}
    \forall t\in \mathbb{R}\forall x\in M:\sigma_t^{\varphi_0}(x)=d_0^{it}xd_0^{-it}.
\end{equation}
We define a unitary operator $u_0$ on the Hilbert space $L^2(\mathbb{R},H)$ by
\begin{equation}
    (u_0\xi)(t)=d_0^{it}\xi(t),\xi\in L^2(\mathbb{R},H),t\in \mathbb{R}.
\end{equation}
Recall that the crossed product $N=R(M,\sigma^{\varphi_0})$ is generated by the elements $\pi(x),x\in M$, and $\lambda(s),s\in \mathbb{R}$, as described in the beginning of Chapter II. We shall describe the action of $u_0(\cdot)u_0^*$ on these generating elements.
% \end{document}