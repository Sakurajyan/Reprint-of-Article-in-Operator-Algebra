% \documentclass[12pt]{report}
% \usepackage{geometry}
%  \geometry{
%  a4paper,
%  total={170mm,257mm},
%  left=20mm,
%  top=20mm,
%  }
% \usepackage[hidelinks]{hyperref}
% \usepackage{tikz-cd}
% \usepackage{amsthm}
% \usepackage{amsmath}
% \usepackage{amssymb}
% \usepackage{amsfonts}
% \usepackage{xcolor}
% \usepackage{physics}
% \usepackage{mathrsfs}
% \usepackage{bbm}

% \tikzcdset{every label/.append style = {font = \normalsize}}
% \newtheorem{theorem}{Theorem}[section]
% \newtheorem{proposition}[theorem]{Proposition}
% \newtheorem{corollary}[theorem]{Corollary}
% \newtheorem{definition}[theorem]{Definition}
% \theoremstyle{definition}
% \newtheorem{example}[theorem]{Example}
% \newcommand{\inner}[1]{\langle#1\rangle}
% \newcommand{\Sp}{\operatorname{Sp}}
% \begin{document}
% \tableofcontents
\chapter{$L^p$ Spaces Associated with a Von Neumann Algebra}
In this chapter, we present Haagerup's theory of $L^p$ spaces associated with a von Neumann algebra.\par 
Let $M$ be a von Newmann algebra and let $\varphi_0$ be a normal faithful semifinite weight on $M$.\par 
We denote by $N$ the crossed product $R(M,\sigma^{\varphi_0})$ of $M$ by the modular automorphism group $\sigma^{\varphi_0}$ associated with $\varphi_0$. Recall [18, Section 3; 8, Section 5] that if $M$ is given on a Hilbert space $H$, then $N$ is the Von Neumann algebra on the Hilbert space $L^2(\mathbb{R},H)$ generated by the operators $\pi(x),x\in M$, and $\lambda(s)$, $s\in \mathbb{R}$, defined by
\begin{equation}
    (\pi(x)\xi)(t)=\sigma_{-t}^{\varphi_0}(x)\xi(t),\xi\in L^2(\mathbb{R},H),t\in \mathbb{R},  
\end{equation}
\begin{equation}
    (\lambda(s)\xi)(t)=\xi(t-s),\xi\in L^2(\mathbb{R},H),t\in \mathbb{R}. 
\end{equation}
We identify $M$ with its image $\pi(M)$ in $N$ (recall that $\pi$ normal faithful representation of $M$).\par 
We denote by $\theta$ the dual action of $\mathbb{R}$ in $N$. The $\theta_s$, $s\in \mathbb{R}$, are automorphisms of $N$ characterized by
\begin{equation}\label{Chap2: eqn: 3}
    \theta_s x=x, x\in M
\end{equation} 
\begin{equation}
    \theta_s\lambda(t)=e^{-ist}\lambda(t),t\in \mathbb{R}.
\end{equation}
By \eqref{Chap2: eqn: 3}, $M$ is contained in the set of fixed points under $\theta$. Actually
\begin{equation}\label{Chap2: eqn: 5}
    M=\{y\in N|\forall s\in \mathbb{R}:\theta_s y=y\} 
\end{equation}
(see e.g. [5, Lemma 3.6]).\par 
The $\theta_s$, $s\in \mathbb{R}$, naturally extend to automorphisms, still denoted $\theta_s$, of $\hat{N}_+$, the extended positive part of $N$ [7, Section 1]. 
Recall [8, Lemma 5.2] that the formula 
\begin{equation}
    Tx=\int_{\mathbb{R}}\theta_s(x)\dd s,x\in N_+,
\end{equation}
defines a normal faithful semifinite operator valued weight $T$ from $N$ to $M$ in the following sense: for each $x\in N_+$, $Tx$ is the element of $\hat{N}_+$ characterized by 
\begin{equation}
    \inner{Tx,\chi}=\int_\mathbb{R} \inner{\theta_s(x),\chi}\dd s
\end{equation}
for all $x\in N_*^+$. Note that 
\begin{equation}\label{Chap2: eqn: 8}
    \forall s\in \mathbb{R}: \theta_s\circ T=T.
\end{equation}
In view of \eqref{Chap2: eqn: 5}, this formula implies that the values of $T$ are actually in $\hat{M}_+$.\par 
For each normal weight $\varphi$ on $M$, we put 
\begin{equation}\label{Chap2: eqn: 9}
    \tilde{\varphi}=\hat{\varphi}\circ T  
\end{equation}
where $\hat{\varphi}$ denotes the extension of $\varphi$ to a normal weight on $\hat{M}_+$ as described in [7, Proposition 1.10]. Then $\tilde{\varphi}$ is a normal weight on $N$ [7,Proposition 2.3]; $\tilde{\varphi}$ is called the dual weight of $\varphi$ (see [6, Introduction + Section 1)]. Note that \eqref{Chap2: eqn: 8} and \eqref{Chap2: eqn: 9} imply 
\begin{equation}
    \forall s\in \mathbb{R}: \tilde{\varphi}\circ \theta_s=\tilde{\varphi}.  
\end{equation}
If $\varphi$ and $\psi$ are normal faithful semifinite weights, then so are $\tilde{\varphi}$ and $\tilde{\psi}$, and we have [7, Theorem 4.7]: 
\begin{equation}
    \forall t\in \mathbb{R}\forall x\in M: \sigma_t^{\tilde{\varphi}}(x)=\sigma_t^{\varphi}(x),
\end{equation}
\begin{equation}
    \forall t\in \mathbb{R}:(D\tilde{\varphi}:D\tilde{\psi})_t=(D\varphi:D\psi)_t.
\end{equation}
\begin{lemma}
    1) The mapping 
\[
    \varphi\mapsto \tilde{\varphi}  
\]
    is a bijection of the set of all normal semifinite weights on $M$ onto the set of normal semifinite weights $\psi$ on $N$ satisfying 
    \begin{equation}\label{Chap2: eqn: 13}
        \forall s\in \mathbb{R}:\psi\circ \theta_s=\psi.
    \end{equation}
    2) For all normal weights $\varphi$ and $\psi$ on $M$ and all $x\in M$, we have 
    \begin{enumerate}
        \item $(\varphi+\psi)^\sim=\tilde{\varphi}+\tilde{\psi}$,
        \item $(x\cdot \varphi\cdot x^*)^\sim=x\cdot \tilde{\varphi}\cdot x^*$,
        \item $\supp \tilde{\varphi}=\supp \varphi$.
    \end{enumerate}
\end{lemma}
\begin{proof}
    That $\tilde{\varphi}$ is semifinite if $\varphi$ is follows from the proof of [7, Proposition 2.3]. That $\varphi\mapsto \tilde{\varphi}$ is injective follows from the formula 
\[
    \varphi(\dot{T}x)=\tilde{\varphi}(x),x\in m_T,
\]  
and the fact that $\dot{T}(m_T)$ is $\sigma$-weakly dense in $M$ [7, Proposition 2.5]. \par
Now let us prove 2). First observe that $(\varphi+\psi)^\wedge=\hat{\varphi}+\hat{\psi}$ since $\hat{\varphi}+\hat{\psi}:\hat{M}\to [0,\infty]$ obviously satisfies the properties that characterize $(\varphi+\psi)^\wedge$ ([7, Proposition 1.10]); (a)
 follows trivially. Similarly, $(x\cdot \varphi\cdot x^*)^\wedge=x\cdot \hat{\varphi}\cdot x^*$, whence (b).\par
 To prove (c), put $p_0=1-\supp \varphi$. Then $Mp_0$ is the $\sigma$-weak closure in $M$ of $N_\varphi=\{x\in M|\varphi(x^*x)=0\}$. Similarly, the $\sigma$-weak closure in $N$ of $N_{\tilde{\varphi}}=\{y\in N|\tilde{\varphi}(y^*y)=0\}$ is $Nq_0$ where $q_0=1-\supp \tilde{\varphi}$. Now
\[
    n_TN_\varphi\subset N_{\tilde{\varphi}}  
\] 
since 
 \[
 \begin{split}
     \forall y\in n_T\forall x\in N_\varphi:&\tilde{\varphi}(x^*y^*yx)=\varphi(T(x^*y^*yx))\\
     =&\varphi(x^*T(y^*y)x)\leq \norm{T(y^*y)}\varphi(x^*x)=0.
 \end{split}    
 \]
 As $n_T$ is $\sigma$-weakly dense in $N$, it follows that 
 \[
    N_\varphi\subset \overline{N_{\tilde{\varphi}}}^{\sigma-w}  
 \]
 whence 
 \[
   p_0\leq q_0.  
 \]
 Note that we must have $q_0\in M$ since $\tilde{\varphi}$, and hence $\supp \tilde{\varphi}$, is $\theta$-invariant. Thus to conclude that $p_0=q_0$ we need only show that $\varphi(q_0)=0$. This follows from 
 \[
     \forall x\in m_T:\varphi(q_0\dot{T}(x)q_0)=\varphi(\dot{T}(q_0xq_0))=\tilde{\varphi}(q_0xq_0)=0
 \]
 and the fact that $\dot{T}(m_T)$ is $\sigma$-weakly dense in $M$ [7, Proposition 2.5].\par
 We now return to 1). Let $\psi$ be a normal semifinite weight on 
$N$ satisfying \eqref{Chap2: eqn: 13}. First suppose that $\psi$ is also faithful. Then by [5, (proof of) Theorem 3.7), it follows that $\psi=\tilde{\varphi}$ for some normal faithful semifinite $\varphi$ on $M$.\par 
 In the general case, put $q_0=1-\supp \psi$. Then by \eqref{Chap2: eqn: 13} and \eqref{Chap2: eqn: 5}, we have $q_0\in M$. Now take any normal semifinite weight $\chi_0$ on $M$ 
\end{proof}
% \end{document}