% \documentclass[12pt]{report}
% \usepackage{geometry}
%  \geometry{
%  a4paper,
%  total={170mm,257mm},
%  left=20mm,
%  top=20mm,
%  }
% \usepackage[hidelinks]{hyperref}
% \usepackage{tikz-cd}
% \usepackage{amsthm}
% \usepackage{amsmath}
% \usepackage{amssymb}
% \usepackage{amsfonts}
% \usepackage{xcolor}
% \usepackage{physics}
% \usepackage{mathrsfs}
% \usepackage{bbm}

% \tikzcdset{every label/.append style = {font = \normalsize}}
% \newtheorem{theorem}{Theorem}[section]
% \newtheorem{proposition}[theorem]{Proposition}
% \newtheorem{corollary}[theorem]{Corollary}
% \newtheorem{definition}[theorem]{Definition}
% \theoremstyle{definition}
% \newtheorem{example}[theorem]{Example}
% \newcommand{\inner}[1]{\langle#1\rangle}
% \newcommand{\Sp}{\operatorname{Sp}}
% \begin{document}
% \tableofcontents
\chapter{$L^p$ Spaces Associated with a Von Neumann Algebra}
In this chapter, we present Haagerup's theory of $L^p$ spaces associated with a von Neumann algebra.\par 
Let $M$ be a von Newmann algebra and let $\varphi_0$ be a normal faithful semifinite weight on $M$.\par 
We denote by $N$ the crossed product $R(M,\sigma^{\varphi_0})$ of $M$ by the modular automorphism group $\sigma^{\varphi_0}$ associated with $\varphi_0$. Recall [18, Section 3; 8, Section 5] that if $M$ is given on a Hilbert space $H$, then $N$ is the Von Neumann algebra on the Hilbert space $L^2(\mathbb{R},H)$ generated by the operators $\pi(x),x\in M$, and $\lambda(s)$, $s\in \mathbb{R}$, defined by
\begin{equation}
    (\pi(x)\xi)(t)=\sigma_{-t}^{\varphi_0}(x)\xi(t),\xi\in L^2(\mathbb{R},H),t\in \mathbb{R},  
\end{equation}
\begin{equation}
    (\lambda(s)\xi)(t)=\xi(t-s),\xi\in L^2(\mathbb{R},H),t\in \mathbb{R}. 
\end{equation}
We identify $M$ with its image $\pi(M)$ in $N$ (recall that $\pi$ normal faithful representation of $M$).\par 
We denote by $\theta$ the dual action of $\mathbb{R}$ in $N$. The $\theta_s$, $s\in \mathbb{R}$, are automorphisms of $N$ characterized by
\begin{equation}\label{Chap2: eqn: 3}
    \theta_s x=x, x\in M
\end{equation} 
\begin{equation}
    \theta_s\lambda(t)=e^{-ist}\lambda(t),t\in \mathbb{R}.
\end{equation}
By \eqref{Chap2: eqn: 3}, $M$ is contained in the set of fixed points under $\theta$. Actually
\begin{equation}\label{Chap2: eqn: 5}
    M=\{y\in N|\forall s\in \mathbb{R}:\theta_s y=y\} 
\end{equation}
(see e.g. [5, Lemma 3.6]).\par 
The $\theta_s$, $s\in \mathbb{R}$, naturally extend to automorphisms, still denoted $\theta_s$, of $\hat{N}_+$, the extended positive part of $N$ [7, Section 1]. 
Recall [8, Lemma 5.2] that the formula 
\begin{equation}
    Tx=\int_{\mathbb{R}}\theta_s(x)\dd s,x\in N_+,
\end{equation}
defines a normal faithful semifinite operator valued weight $T$ from $N$ to $M$ in the following sense: for each $x\in N_+$, $Tx$ is the element of $\hat{N}_+$ characterized by 
\begin{equation}
    \inner{Tx,\chi}=\int_\mathbb{R} \inner{\theta_s(x),\chi}\dd s
\end{equation}
for all $x\in N_*^+$. Note that 
\begin{equation}\label{Chap2: eqn: 8}
    \forall s\in \mathbb{R}: \theta_s\circ T=T.
\end{equation}
In view of \eqref{Chap2: eqn: 5}, this formula implies that the values of $T$ are actually in $\hat{M}_+$.\par 
For each normal weight $\varphi$ on $M$, we put 
\begin{equation}\label{Chap2: eqn: 9}
    \tilde{\varphi}=\hat{\varphi}\circ T  
\end{equation}
where $\hat{\varphi}$ denotes the extension of $\varphi$ to a normal weight on $\hat{M}_+$ as described in [7, Proposition 1.10]. Then $\tilde{\varphi}$ is a normal weight on $N$ [7,Proposition 2.3]; $\tilde{\varphi}$ is called the dual weight of $\varphi$ (see [6, Introduction + Section 1)]. Note that \eqref{Chap2: eqn: 8} and \eqref{Chap2: eqn: 9} imply 
\begin{equation}
    \forall s\in \mathbb{R}: \tilde{\varphi}\circ \theta_s=\tilde{\varphi}.  
\end{equation}
If $\varphi$ and $\psi$ are normal faithful semifinite weights, then so are $\tilde{\varphi}$ and $\tilde{\psi}$, and we have [7, Theorem 4.7]: 
\begin{equation}
    \forall t\in \mathbb{R}\forall x\in M: \sigma_t^{\tilde{\varphi}}(x)=\sigma_t^{\varphi}(x),
\end{equation}
\begin{equation}
    \forall t\in \mathbb{R}:(D\tilde{\varphi}:D\tilde{\psi})_t=(D\varphi:D\psi)_t.
\end{equation}
\begin{lemma}\label{Chap2: lemma: 1}
    1) The mapping 
\[
    \varphi\mapsto \tilde{\varphi}  
\]
    is a bijection of the set of all normal semifinite weights on $M$ onto the set of normal semifinite weights $\psi$ on $N$ satisfying 
    \begin{equation}\label{Chap2: eqn: 13}
        \forall s\in \mathbb{R}:\psi\circ \theta_s=\psi.
    \end{equation}
    2) For all normal weights $\varphi$ and $\psi$ on $M$ and all $x\in M$, we have 
    \begin{enumerate}
        \item $(\varphi+\psi)^\sim=\tilde{\varphi}+\tilde{\psi}$,
        \item $(x\cdot \varphi\cdot x^*)^\sim=x\cdot \tilde{\varphi}\cdot x^*$,
        \item $\supp \tilde{\varphi}=\supp \varphi$.
    \end{enumerate}
\end{lemma}
\begin{proof}
    That $\tilde{\varphi}$ is semifinite if $\varphi$ is follows from the proof of [7, Proposition 2.3]. That $\varphi\mapsto \tilde{\varphi}$ is injective follows from the formula 
\[
    \varphi(\dot{T}x)=\tilde{\varphi}(x),x\in m_T,
\]  
and the fact that $\dot{T}(m_T)$ is $\sigma$-weakly dense in $M$ [7, Proposition 2.5]. \par
Now let us prove 2). First observe that $(\varphi+\psi)^\wedge=\hat{\varphi}+\hat{\psi}$ since $\hat{\varphi}+\hat{\psi}:\hat{M}\to [0,\infty]$ obviously satisfies the properties that characterize $(\varphi+\psi)^\wedge$ ([7, Proposition 1.10]); (a)
 follows trivially. Similarly, $(x\cdot \varphi\cdot x^*)^\wedge=x\cdot \hat{\varphi}\cdot x^*$, whence (b).\par
 To prove (c), put $p_0=1-\supp \varphi$. Then $Mp_0$ is the $\sigma$-weak closure in $M$ of $N_\varphi=\{x\in M|\varphi(x^*x)=0\}$. Similarly, the $\sigma$-weak closure in $N$ of $N_{\tilde{\varphi}}=\{y\in N|\tilde{\varphi}(y^*y)=0\}$ is $Nq_0$ where $q_0=1-\supp \tilde{\varphi}$. Now
\[
    n_TN_\varphi\subset N_{\tilde{\varphi}}  
\] 
since 
 \[
 \begin{split}
     \forall y\in n_T\forall x\in N_\varphi:&\tilde{\varphi}(x^*y^*yx)=\varphi(T(x^*y^*yx))\\
     =&\varphi(x^*T(y^*y)x)\leq \norm{T(y^*y)}\varphi(x^*x)=0.
 \end{split}    
 \]
 As $n_T$ is $\sigma$-weakly dense in $N$, it follows that 
 \[
    N_\varphi\subset \overline{N_{\tilde{\varphi}}}^{\sigma-w}  
 \]
 whence 
 \[
   p_0\leq q_0.  
 \]
 Note that we must have $q_0\in M$ since $\tilde{\varphi}$, and hence $\supp \tilde{\varphi}$, is $\theta$-invariant. Thus to conclude that $p_0=q_0$ we need only show that $\varphi(q_0)=0$. This follows from 
 \[
     \forall x\in m_T:\varphi(q_0\dot{T}(x)q_0)=\varphi(\dot{T}(q_0xq_0))=\tilde{\varphi}(q_0xq_0)=0
 \]
 and the fact that $\dot{T}(m_T)$ is $\sigma$-weakly dense in $M$ [7, Proposition 2.5].\par
 We now return to 1). Let $\psi$ be a normal semifinite weight on 
$N$ satisfying \eqref{Chap2: eqn: 13}. First suppose that $\psi$ is also faithful. Then by [5, (proof of) Theorem 3.7), it follows that $\psi=\tilde{\varphi}$ for some normal faithful semifinite $\varphi$ on $M$.\par 
 In the general case, put $q_0=1-\supp \psi$. Then by \eqref{Chap2: eqn: 13} and \eqref{Chap2: eqn: 5}, we have $q_0\in M$. Now take any normal semifinite weight $\chi_0$ on $M$ such that $\supp\chi_0=q_0$. Then $\widetilde{\chi}_0$ is a normal faithful semifinite $\theta$-invariant weight on $N$ with $\supp \widetilde{\chi}_0=q_0$. Hence $\widetilde{\chi}_0+\psi$ is faithful and thus, as above, 
 \[
    \widetilde{\chi}_0+\psi=\tilde{\varphi}
 \]
 for some normal faithful semifinite weight $\varphi$ on $M$. Finally, using (b), we find that 
\[
   \begin{split}
       \psi=&(1-q_0)\cdot (\widetilde{\chi}_0+\psi)\cdot (1-q_0)\\
       =&(1-q_0)\cdot \tilde{\varphi}\cdot (1-q_0)\\
       =&((1-q_0)\cdot \varphi\cdot (1-q_0))^\sim.
   \end{split} 
\]
\end{proof}

Denote by $\tau$ the normal faithful semifinite trace on $N$ characterized by 
\begin{equation}
    \forall t\in \mathbb{R}:(D\tilde{\varphi}_0:D\tau)_t=\lambda(t)
\end{equation} 
(for the existence, see [8, Lemma 5.2]); $\tau$ satisfies 
\begin{equation}\label{Chap2: eqn: 15}
    \forall s\in \mathbb{R}:\tau\circ \theta_s=e^{-s}\tau.
\end{equation}
With each $h\in \hat{N}_+$ we associate the normal weight $\tau(h\cdot)$ on $N$ as in [8, remarks preceding Proposition 1.11]. When $h$ is simply a positive self-adjoint operator affiliated with $N$ (see [7, Example 1.2]), this definition agrees with that given in [14, Section 4].\par 
We recall some facts about the mapping $h\mapsto \tau(h\cdot)$ (see [7, Theorem 1.12 (and its proof) and Preposition 1.11, (4)]):
\begin{lemma}\label{Chap2: lemma: 2}
    1) The mapping 
\[
    h\mapsto \tau(h\cdot)  
\]
is a bijection of $\hat{N}_+$ onto the set of normal weights on $N$. In particular, it is a bijection of the positive self-adjoint operators affiliated with $N$ onto the normal semifinite weights on $N$.\par 
2) For all $h,k\in \hat{N}_+$ and all $x\in N$, we have
\begin{enumerate}
    \item $\tau((h\dot{+}k)\cdot)=\tau(h\cdot)+\tau(k\cdot)$,
    \item $\tau((x\cdot h\cdot x^*)\cdot)=x\cdot \tau(h\cdot)\cdot x^*$,
    \item $\supp \tau(h\cdot)=\supp h$.
\end{enumerate} 
\end{lemma} 
Here, $h\dot{+}k$ and $x\cdot h\cdot x^*$ denote the operations in $\hat{N}_+$ introduced in [7, Definition 1.3]. If $h$ and $k$ are positive self-adjoint operators such that $D(h^\frac{1}{2})\cap D(k^\frac{1}{2})$ is dense, then $h\dot{+}k$ is the simply the form sum of $h$ and $k$ [2, Corollary 4.13]. If $h$ is a positive self-adjoint operator and $x$ a bounded operator such that $D(h^\frac{1}{2}x^*)$ is dense, then $x\cdot h\cdot x^*=\abs{h^\frac{1}{2}x^*}^2$.\par 
\begin{definition}
    For each normal weight $\varphi$ on $M$ we define $h_\varphi$ as the unique element of $\hat{N}_+$ given by 
   \begin{equation}
       \tilde{\varphi}=\tau(h_\varphi\cdot).
   \end{equation}
\end{definition}
\begin{proposition}\label{Chap2: Prop: 4}
    1) The mapping 
\[
    \varphi\mapsto h_\varphi
\]
is a bijection of the set of all normal semifinite weights on $M$ onto the set of all positive self-adjoint operators $h$ affiliated with $N$ satisfying
\begin{equation}\label{Chap2: eqn: 17}
    \forall s\in \mathbb{R}:\theta_sh=e^{-s}h.
\end{equation}
(2) For all normal weights $\varphi$ and $\psi$ on $M$ and all $x\in M$, we have
\begin{enumerate}
    \item $h_{\varphi+\psi}+h_\varphi\dot{+}h_\psi$,
    \item $h_{x\cdot \varphi \cdot x^*}=x\cdot h_\varphi\cdot x^*$,
    \item $\supp h_\varphi=\supp\varphi$.
\end{enumerate}
\end{proposition}
\begin{proof}
    This proposition is an immediate consequence of Lemma \ref{Chap2: lemma: 1} and \ref{Chap2: lemma: 2}. We just need to prove that a positive self-adjoint operator $h$ affiliated with $N$ satisfies \eqref{Chap2: eqn: 17} if and only if the corresponding weight $\tau(h\cdot)$ is $\theta$-invariant. This follows easily from \eqref{Chap2: eqn: 15}. Indeed, for all $s\in \mathbb{R}$ we have
    \[
        \tau(e^s\theta_s(h)\cdot)=e^s(\tau\circ \theta_s)(h\theta_{-s}(\cdot))=\tau(h\theta_{-s}(\cdot))=\tau(h\cdot )\circ \theta_{-s},
    \]
    whence 
    \[
      e^s\theta_s(h)=h\Leftrightarrow \tau(e^s\theta_s(h)\cdot)=\tau(h\cdot)\Leftrightarrow \tau(h\cdot)=\tau(h\cdot)\circ \theta_{-s}.  
    \]
    The equivalence of \eqref{Chap2: eqn: 17} and
    \[
        \forall s\in \mathbb{R}:\tau(h\cdot)=\tau(h\cdot)\circ \theta_s    
    \]
    follows.
\end{proof}
The next lemma is essential. It will permit us apply results on $\tau$-measurable operators.\par
As usual, $\chi_{]\gamma,\infty[}$ denotes the characteristic function for the interval $]\gamma,\infty[$.
\begin{lemma}\label{Chap2: Lemma: 5}
    Let $\varphi$ be a normal semifinite weight on $M$. Then for all $\gamma\in \mathbb{R}_+$, we have 
    \[
      \tau(\chi_{]\gamma,\infty[}(h_\varphi))=\frac{1}{\gamma}\varphi(1).
    \]
\end{lemma}
\begin{proof}
    First let us prove the formula in the case $\gamma=1$.\par 
Let $s\in \mathbb{R}$. Then since $\theta_s$ is an automorphism and $\theta_sh_\varphi=e^{-s}h_\varphi$ we have
\[
    \theta_s(h_\varphi^{-1}\chi_{]1,\infty[}(h_\varphi))=e^sh_\varphi^{-1}\chi_{]1,\infty[}(e^{-s}h_\varphi).
\]
Now let $h_\varphi=\int\lambda\dd e_\lambda$ be the spectral decomposition of $h_\varphi$. Then for any vector functional $\omega_{\xi,\xi}$, where $\xi$ is a unit vector, we have 
\[
    \begin{split}
        \inner{\int_\mathbb{R}\theta_s(h_\varphi^{-1}\chi_{]1,\infty[}(h_\varphi))\dd s,\omega_{\xi,\xi}}=&\int_\mathbb{R}\inner{e^sh_\varphi^{-1}\chi_{]1,\infty[}(e^{-s}h_\varphi),\omega_{\xi,\xi}}\dd s\\ 
        =&\int_\mathbb{R}\int_{]0,\infty[}e^s\lambda^{-1}\chi_{]1,\infty[}(e^{-s}\lambda)\dd (e_\lambda \xi|\xi)\dd s\\
        =&\int_{]0,\infty[}\lambda^{-1}\left( \int_{]-\infty,\log \lambda[}e^s\dd s \right)\dd (e_\lambda \xi|\xi)\\
        =&\int_{]0,\infty[}\lambda^{-1}\lambda\dd (e_\lambda \xi|\xi)\\
        =&\norm{(\supp h_\varphi)\xi}^2
    \end{split}
\]
So that
\[
    \int_\mathbb{R}\theta_s(h_\varphi^{-1}\chi_{]1,\infty[}(h_\varphi))\dd s=\supp h_\varphi=\supp \varphi.
\]
Finally, since $\tilde{\varphi}=\tau(h_\varphi\cdot)$ we have 
\[
\begin{split}
    \tau(\chi_{]1,\infty[}(h_\varphi))=&\tau(h_\varphi^\frac{1}{2}(h_\varphi^{-1}\chi_{]1,\infty[}(h_\varphi))h_\varphi^\frac{1}{2})\\
    =&\tilde{\varphi}(h_\varphi^{-1}\chi_{]1,\infty[}(h_\varphi))\\
    =&\varphi\left( \int \theta_s(h_\varphi^{-1}\chi_{]1,\infty[}(h_\varphi))\dd s \right)=\varphi(\supp \varphi)=\varphi(1).
\end{split}
\]
This completes the proof in the case $\gamma=1$. In the general case, write $\gamma=e^s$, $s\in \mathbb{R}$. Then by \eqref{Chap2: eqn: 15} 
\[
    \begin{split}
        \tau(\chi_{]e^s,\infty[}(h_\varphi))=&\tau(\chi_{]1,\infty[}(e^{-s}h_\varphi))\\
        =&\tau(\theta_s(\chi_{]1,\infty[}(h_\varphi)))\\
        =&e^{-s}\tau(\chi_{]1,\infty[}(h_\varphi))=e^{-s}\varphi(1).\\
    \end{split}  
\]
\end{proof}
By Chapter I, Proposition \ref{prop: 21}, we have 
\begin{corollary}\label{Chap2: Coro: 6}
    Let $\varphi$ be a normal semifinite weight on $M$. Then $h_\varphi$ is $\tau$-measurable if and only if $\varphi\in M_*$. 
\end{corollary}
We denote by $\tilde{N}$ the set of all $\tau$-measurable closed densely defined operators affiliated with $N$. Recall (Chapter I) that $\tilde{N}$ is a topological *-algebra with respect to strong sum and product. Sums and products of elements in $\tilde{N}$ will always be understood to be 
in the strong sense although we do not emphasize it in the notation.\par 
We denote by $\tilde{N}_+$ the subset of all positive self-adjoint elements of $\tilde{N}$.\par
Note that the $\theta_s$, $s\in \mathbb{R}$, extend to continuous *-automorphisms, still denoted $\theta_s$, of $\tilde{N}$. We have 
\begin{equation}
    \forall s\in \mathbb{R} \forall \epsilon,\delta\in \mathbb{R}_+: \theta_s(N(\epsilon,\delta))=N(\epsilon,e^{-s}\delta)
\end{equation}
Since for all $a\in \tilde{N}_+$
\[
    \tau(\chi_{]\epsilon,\infty[}(\theta_sa))=\tau(\theta_s(\chi_{]\epsilon,\infty[}(a)))=e^{-s}\tau(\chi_{]\epsilon,\infty[}(a))  
\]
(for the definition and properties of the $0$-neighbourhoods $N(\epsilon,\delta)$, we refer to Chapter I).
\begin{theorem}\label{Chap2: Thm: 7}
    1) The mapping 
\[
  \varphi\mapsto h_\varphi  
\]
extends to a linear bijection, still denoted $\varphi\mapsto h_\varphi$, of $M_*$ onto the subspace
\begin{equation}\label{Chap2: eqn: 19}
    \{h\in \tilde{N}|\forall s\in \mathbb{R}:\theta_sh=e^{-s}h\}  
\end{equation}
of $N$.\par 
2) For all $\varphi\in M_*$ and $x,y \in M$, we have 
\begin{equation}\label{Chap2: eqn: 20}
    h_{x\cdot \varphi \cdot y^*}=x h_\varphi y^*  
\end{equation}
and
\begin{equation}\label{Chap2: eqn: 21}
     h_{\varphi^*}=h_\varphi^*.  
\end{equation}
3) If $\varphi=u\abs{\varphi}$ is the polar decomposition of $\varphi$, then $h=uh_{\abs{\varphi}}$ {\color{red} ($h_\varphi=uh_{\abs{\varphi}}$)} is the polar decomposition of $h_\varphi$. In particular, 
\begin{equation}
    \abs{h_\varphi}=h_{\abs{\varphi}}.
\end{equation}
\end{theorem}
The proof will be based on Corollary \ref{Chap2: Coro: 6}, Proposition \ref{Chap2: Prop: 4}, and the following lemma.
\begin{lemma}\label{Chap2: lemma: 8}
    1) Let $h$ and $k$ be positive self-adjoint operators such that $D(h^\frac{1}{2})\cap D(k^\frac{1}{2})$ is dense. Then 
    \[
       h+k\subset h\dot{+}k. 
    \]
    If $h+k$ is essentially self-adjoint, then its unique self-adjoint extension is precisely $h\dot{+}k$.\par 
2) Let $h$ be a positive self-adjoint operator and $x$ a bounded operator such that $D(h^\frac{1}{2}x^*)$ is dense. Then 
\[
    xhx^*\subset x\cdot h\cdot x^*.
\]
If $xhx^*$ is essentially self-adjoint, then its unique self-adjoint extension is precisely $x\cdot h\cdot x^*$. 
\end{lemma}
\begin{proof}
    1) Recall that by definition $h\dot{+}k$ is the unique positive self-adjoint operator characterized by $D((h\dot{+}k)^\frac{1}{2})=D(h^\frac{1}{2})\cap D(k^\frac{1}{2})$ and 
    \begin{equation}
        \forall \xi\in D(h^\frac{1}{2})\cap D(k^\frac{1}{2}):\norm{(h\dot{+}k)^\frac{1}{2}\xi}^2=\norm{h^\frac{1}{2}\xi}^2+\norm{k^\frac{1}{2}\xi}^2.
    \end{equation}
    By polarization, it follows that 
\[
    \forall \xi\in D(h^\frac{1}{2})\cap D(k^\frac{1}{2}):((h\dot{+}k)^\frac{1}{2}\xi|(h\dot{+}k)^\frac{1}{2}\eta)=(h^\frac{1}{2}\xi|h^\frac{1}{2}\eta)+(k^\frac{1}{2}\xi|k^\frac{1}{2}\eta).
\]
    Now let $\xi\in D(h+k)=D(h)\cap D(k)$ and $\eta\in D(h\dot{+}k)$. Then also $\xi\in D(h^\frac{1}{2})\cap D(k^\frac{1}{2})$ and $\eta\in D((h\dot{+}k)^\frac{1}{2})=D(h^\frac{1}{2})\cap D(k^\frac{1}{2})$ so that 
    \[
       \begin{split}
           ((h+k)\xi|\eta)=&(h\xi|\eta)+(k\xi|\eta)\\
            =&(h^\frac{1}{2}\xi|h^\frac{1}{2}\eta)+(k^\frac{1}{2}\xi|k^\frac{1}{2}\xi)\\
            =&((h\dot{+}k)^\frac{1}{2}\xi|(h\dot{+}k)^\frac{1}{2}\eta)\\
            =&(\xi|(h\dot{+}k)\eta).
       \end{split}
    \]
    It follows that 
    \[
        h+k\subset (h\dot{+}k)^*=(h\dot{+}k).
    \]
    Hence $h+k$ is preclosed and $[h+k]\subset h\dot{+}k$. If $[h+k]$ is self-adjoint, we must have $[h+k]=h\dot{+}k$.\par 
    2) Recall that $x\cdot h\cdot x^*=\abs{h^\frac{1}{2}x^*}^2$. Now let $\xi\in D(xhx^*)=D(hx^*)$ and $\eta\in D(x\cdot h\cdot x^*)$. Then also $\xi\in D(h^\frac{1}{2}x^*)$ and $\eta\in D((x\cdot h\cdot x^*)^\frac{1}{2})=D(h^\frac{1}{2}x^*)$ so that 
    \[
      (xhx^* \xi|\eta)=(hx^*\xi|x^* \eta)=(h^\frac{1}{2}x^*\xi|h^\frac{1}{2}x^*\eta)=(\xi|(x\cdot h\cdot x^*)\eta).  
    \]
    It follows that 
    \[
        xhx^*\subset(x\cdot h\cdot x^*)^*=x\cdot h\cdot x^*.
    \]
    Hence $xhx^*$ is preclosed and $[xhx^*]\subset x\cdot h\cdot x^*$. If $[xhx^*]$ is self-adjoint, we must have $[xhx^*]=x\cdot h\cdot x^*$.
\end{proof}
\begin{proof}[Proof of Theorem \ref{Chap2: Thm: 7}]
    Let $\varphi,\psi\in M_*^+$. Then $h_\varphi$ and $h_\psi$ are positive self-adjoint and $\tau$-measurable so that their strong sum exists and is again a positive self-adjoint $\tau$-measurable operator. By Lemma \ref{Chap2: lemma: 8}, this sum then coincides with $h_\varphi\dot{+}h_\psi$. Then Proposition \ref{Chap2: Prop: 4} yields 
\[
  h_{\varphi+\psi}=h_\varphi+h_\psi,  
\]
where the sum at the right hand side is now the sum in $\tilde{N}$.\par 
Similarly for all $\varphi\in M_*^+$ and $x\in M$ we get 
\begin{equation}\label{Chap2: eqn: 24}
    h_{x\cdot \varphi\cdot \xi^*}=xh_\varphi x^*.
\end{equation}
Now the additive and homogeneous mapping $\varphi\mapsto h_\varphi$ of $M_*^+$ onto $\{h\in \tilde{N}_+|\forall s\in \mathbb{R}:\theta_sh=e^{-s}h\}$ extends by linearity to a linear mapping $\varphi\mapsto h_\varphi$ of $M_*$ onto the subspace of $\tilde{N}$ spanned by $\{h\in \tilde{N}_+|\forall s\in \mathbb{R}:\theta_sh=e^{-s}h\}$, i.e. onto the subspace \eqref{Chap2: eqn: 19} (evidently, \eqref{Chap2: eqn: 19} is stable under $h\mapsto h^*$ and $h\mapsto\abs{h}$ and hence spanned by its positive elements).\par 
By linearity, we must have \eqref{Chap2: eqn: 21} for a11 $\varphi\in M_*$. Also by linearity, \eqref{Chap2: eqn: 24} holds for all $\varphi\in M_*$ and $x\in M$; by polarization the equation \eqref{Chap2: eqn: 20} follows for all $\varphi\in M_*$ and $x,y\in M$.\par 
In particular, if $\varphi\in u\abs{\varphi}$ is the polar decomposition of $\varphi$, we have 
\[
    h_\varphi=h_{u\abs{\varphi}}=uh_{\abs{\varphi}}.
\]
That this relation is the polar decomposition of $h_\varphi$ follows from the fact that the initial projection for the partial isometry $u$ is $\supp \abs{\varphi}=\supp h_\abs{\varphi}$.\par 
Finally, $\varphi\mapsto h_\varphi$ is injective: if $h_\varphi=0$, then $h_{\abs{\varphi}}=\abs{h_\varphi}=0$, whence $\abs{\varphi}=0$ and $\varphi=0$.
\end{proof}
Motivated by Theorem \ref{Chap2: Thm: 7}, we now give the following definition:
\begin{definition}\label{Chap2: Def: 9}
    For each $p\in [1,\infty]$, we let
    \[
        L^p(M)=\{a\in \tilde{N}|\forall s\in \mathbb{R}:\theta_s a=e^{-\frac{s}{p}}a\}.
    \] 
\end{definition}
Note that the $L^p(M)$ are linear subspaces of $\tilde{N}$ and that they are linearly spanned by their positive part $L^p(M)_+=L^p(M)\cap \tilde{N}_+$.\par 
By Theorem \ref{Chap2: Thm: 7}, we know that $L^1(M)\cong M_*$. And: 
\begin{proposition}\label{Chap2: Prop: 10}
    We have $L^\infty(M)=M$.
\end{proposition}
\begin{proof}
    In view of \eqref{Chap2: eqn: 5}, we just need to show that every $a\in L^\infty(M)$ is bounded. Let $a\in L^\infty(M)$. Then for all $s\in \mathbb{R}$ and all $\lambda\in \mathbb{R}_+$ we have
    \[
        \begin{split}
            \tau(\chi_{]\lambda,\infty[}(\abs{a}))=&\tau(\chi_{]\lambda,\infty[}(\theta_s\abs{a}))\\
            =&\tau(\theta_s(\chi_{]\lambda,\infty[}(\abs{a})))=e^{-s}\tau(\chi_{]\lambda,\infty[}(\abs{a})).
        \end{split}
    \]
    Hence for all $\lambda\in \mathbb{R}_+$ we must have 
    \[
        \tau(\chi_{]\lambda,\infty[}(\abs{a}))=0 \text{ or } \tau(\chi_{]\lambda,\infty[}(\abs{a}))=\infty.
    \]
Since $a$ is $\tau$-measurable, we have $\tau(\chi_{]\lambda,\infty[}(\abs{a}))<\infty$ for some $\lambda$. Hence $\tau(\chi_{]\lambda,\infty[}(\abs{a}))=0$ and thus $\chi_{]\lambda,\infty[}(\abs{a})=0$ since $\tau$ is faithful. This means that $a$ is bounded.
\end{proof}
\begin{remark}
    We have seen that all elements of $L^\infty(M)$ are bounded. In contrast to this, all non-zero elements of $L^p(M)$, where $p<\infty$, are unbounded. To see this, let $a\in L^p(M)$ and suppose that $a\neq 0$. Then for some $\lambda\in \mathbb{R}_+$ we have $\chi_{]\lambda,\infty[}(\abs{a})\neq 0$ and hence $\tau(\chi_{]\lambda,\infty[}(\abs{a}))\neq 0$. Then for all $\mu\in \mathbb{R}_+$ we have 
    \[
        \tau(\chi_{]\mu,\infty[}(\abs{a}))\neq 0
    \]
    since for all $s\in \mathbb{R}$
    \[
       \begin{split}
           \tau(\chi_{]e^\frac{s}{p}\lambda,\infty[}(\abs{a}))=&\tau(\chi_{]\lambda,\infty[}(e^{-\frac{s}{p}}\abs{a}))\\
           =&\tau(\chi_{]\lambda,\infty[}(\theta_s\abs{a}))\\
           =&\tau(\theta_s\chi_{]\lambda,\infty[}(\abs{a}))\\
           =&e^{-s}\tau(\chi_{]\lambda,\infty[}(\abs{a}))\neq 0.
       \end{split} 
    \]
    It follows that $\abs{a}$ must be unbounded.
\end{remark}
\begin{proposition}
    Let $a$ be a closed densely defined operator affiliated with $N$ with polar decomposition $a=u\abs{a}$. Let $p\in [1,\infty[$. Then
    \[
        a\in L^p(M)
    \] 
    if and only if 
    \[
        u\in M \text{ and } \abs{a}^p\in L^1(M).
    \]
\end{proposition}
\begin{proof}
    Recall that $a\in \tilde{N}$ if and only if $\abs{a}\in\tilde{N}$. Furthermore, $\abs{a}\in\tilde{N}$ if and only if $\abs{a}^p\in\tilde{N}$ since $\tau(\chi_{]\lambda,\infty[}(\abs{a}))=\tau(\chi_{]\lambda^p,\infty[}(\abs{a}^p))$ for all $\lambda\in \mathbb{R}_+$. For all such $a$ and all $s\in \mathbb{R}$ we have 
    \[
        \theta_sa=e^{-\frac{s}{p}}a\Leftrightarrow \theta_su=u \text{ and }\theta_s\abs{a}^p=e^{-s}\abs{a}^p.
    \]
    The result follows by Definition \ref{Chap2: Def: 9} and Proposition \ref{Chap2: Prop: 10}.
\end{proof}
 A similar result holds for the right polar decomposition. 
\begin{definition}\label{Chap2: Def: 13}
    We define a linear functional $\tr$ on $L^1(M)$ by
    \[
       \tr(h_\varphi)=\varphi(1),\varphi\in M_*. 
    \]
\end{definition}
Note that 
\begin{equation}
    \tr(\abs{h_\varphi})=\tr(h_{\abs{\varphi}})=\abs{\varphi}(1)=\norm{\varphi}
\end{equation}
for all $\varphi\in M_*$. This implies that 
\begin{equation}\label{Chap2: eqn: 26}
    \abs{\tr(a)}\leq\tr(\abs{a})
\end{equation}
for all $a\in L^1(M)$ and that the mapping $a\mapsto\tr(\abs{a})$ is a norm on $L^1(M)$. 
\begin{definition}
    Let $p\in [1,\infty[$. Then we define $\norm{\cdot}_p$ on $L^p(M)$ by 
    \[
        \norm{a}_p=\tr(\abs{a}^p)^\frac{1}{p}, a\in L^p(M).  
    \]
For $p=\infty$, we put 
\[
    \norm{a}_\infty=\norm{a}, a\in L^\infty(M).
\]
\end{definition}
We shall see that for all $p$, $\norm{\cdot}_p$ is a norm on $L^p(M)$.\par
By \eqref{Chap2: eqn: 26}, we have 
\begin{proposition}
    The mapping 
    \[
       \varphi\mapsto h_\varphi:M_*\to L^1(M)    
    \]
    is an isometry of $M_*$ onto $L^1(M)$. 
\end{proposition}
\begin{lemma}\label{Chap2: Lemma: 16}
    Let $p\in[1,\infty[$ and $\epsilon,\delta\in \mathbb{R}_+$. Then
    \[
        N(\epsilon,\delta)\cap L^p(M)=\{a\in L^p(M)|\norm{a}_p\leq \epsilon\delta^\frac{1}{p}\}.
    \] 
\end{lemma}
\begin{proof}
    Let $a\in L^p(M)$. Then $\abs{a}^p\in L^1(M)_+$ and hence $\abs{a}^p=h_\varphi$ for some $\varphi\in M_*^+$. Now 
    \[
        \begin{split}
            \tau(\chi_{]\epsilon,\infty[}(\abs{a}))=&\tau(\chi_{]\epsilon^p,\infty[}(\abs{a}^p))\\
            =&\frac{1}{\epsilon^p}\varphi(1)\\
            =&\frac{1}{\epsilon^p}\norm{\abs{a}^p}_1=\frac{1}{\epsilon^p}\norm{a}_p^p
        \end{split}
    \]
Using this we get 
\[
    \begin{split}
        a\in N(\epsilon,\delta)\Leftrightarrow& \abs{a}\in N(\epsilon,\delta)\\
        \Leftrightarrow& \tau(\chi_{]\epsilon,\infty[}(\abs{a}))\leq \delta\\
        \Leftrightarrow& \frac{1}{\epsilon^p}\norm{a}_p^p\leq \delta\\
        \Leftrightarrow& \norm{a}_p\leq \epsilon\delta^\frac{1}{p}.
    \end{split}
\]
\end{proof}
\begin{corollary}\label{Chap2: Coro: 17}
    On $L^1(M)$ the norm topology is exactly the topology induced from $\tilde{N}$.
\end{corollary}
We denote by $\mathbb{C}_+$ the closed half-plane $\{a\in \mathbb{C}|\Re a\geq 0\}$ and by $\mathbb{C}^{\circ}_+$ the corresponding open half-plane.
\begin{lemma}  
    Let $h\in \tilde{N}_+$. Then the mapping 
    \[
        \alpha\mapsto h^\alpha:\mathbb{C}_+^\circ\to \tilde{N}
    \]
    is differentiable. 
\end{lemma}
\begin{proof}
    First note that all $h^\alpha$, $\alpha\in \mathbb{C}_+^\circ$, are actually $\tau$-measurable since $h$ is $\tau$-measurable.\par
1) Suppose that $h$ is bounded, i.e. $h\in N_+$. Then the mapping
\[
    \alpha\mapsto h^\alpha:\mathbb{C}_+^\circ\to N
\] 
is differentiable with respect to the norm topology on $N$ and 
\begin{equation}\label{Chap2: eqn: 27}
    \dv{\alpha}h^\alpha=h^\alpha \log h
\end{equation}
(note that the expression at the right hand side is defined for any positive $h\in N$ since the function $\lambda\mapsto \lambda^\alpha \log \lambda$ is continuous on the closed half-plane $\mathbb{C}_+$). This follows from spectral theory using the fact that for all $\alpha_0\in \mathbb{C}_+^\circ$ we have 
\[
    \begin{split}
        \frac{1}{\alpha-\alpha_0}(\lambda^\alpha-\lambda^{\alpha_0})-\lambda^{\alpha_0}\log \lambda=&\frac{1}{\alpha-\alpha_0}(e^{\alpha\log \lambda}-e^{\alpha_0\log \lambda})-\log \lambda e^{\alpha_0\log \lambda}\\
        \to& 0 \text{ as $\alpha\to \alpha_0$ uniformly in $\lambda\in]0,\norm{h}]$.}
    \end{split}
\]
2) Now let $h$ be any element of $\tilde{N}_+$. We claim that $\alpha\mapsto h^\alpha:\mathbb{C}_+^\circ \to \tilde{N}$ is differentiable with respect to the topology on $\tilde{N}$ and that \eqref{Chap2: eqn: 27} still holds (as above, $h^\alpha\log h$ is a well-defined positive self-adjoint operator and, by spectral theory, it is $\tau$-measurable). Now let $\epsilon,\delta\in \mathbb{R}_+$. Take $\lambda\in \mathbb{R}_+$ such that $\tau(\chi_{]\lambda,\infty[}(h))\leq \delta$. Put $p=\chi_{[0,\lambda]}(h)$. Then $hp$ is bounded and by the first part of the proof 
\[
    \begin{split}
        &\norm{\left( \frac{1}{\alpha-\alpha_0}(h^\alpha-h^{\alpha_0})-h^{\alpha_0}\log h \right)p}\\
        =&\norm{ \frac{1}{\alpha-\alpha_0}((hp)^\alpha-(hp)^{\alpha_0})-(hp)^{\alpha_0}\log (hp) }\leq \epsilon
    \end{split}
\]
{\color{red} Origin article here is $(hp)^{\alpha}\log (hp)$} for all $\alpha\in \mathbb{C}_+^\circ$ sufficiently close to $\alpha_0$. Thus
\[
     \frac{1}{\alpha-\alpha_0}(h^\alpha-h^{\alpha_0})-h^{\alpha_0}\log h \in N(\epsilon,\delta)
\] 
for $\alpha$ sufficiently close to $\alpha_0$. This proves the lemma.
\end{proof}
We denote by $S$ the closed complex strip $\{\alpha\in \mathbb{C}|0\leq \Re \alpha\leq 1\}$ and by $S^\circ$ the corresponding open strip. \par
\begin{lemma}\label{Chap2: Lemma: 19}
    Let $h,k\in L^1(M)_+$. Then for $\alpha\in S^\circ$ we have 
\[
    h^\alpha k^{1-\alpha}\in L^1(M),
\]
and the mapping 
\begin{equation}\label{Chap2: eqn: 28}
    \alpha\mapsto h^\alpha k^{1-\alpha}: S^\circ \to L^1(M)
\end{equation} 
is analytic.
\end{lemma}
\begin{proof}
    That $ h^\alpha k^{1-\alpha}\in L^1(M)$ follows from Definition \ref{Chap2: Def: 9} since 
    \[
        \begin{split}
            \forall s\in \mathbb{R}: \theta_s(h^\alpha k^{1-\alpha})=&(\theta_s h)^\alpha (\theta_s k)^{1-\alpha}\\
            =& e^{-\alpha s}h^\alpha e^{-(1-\alpha)s}k^{1-\alpha}=e^{-s}h^\alpha k^{1-\alpha}.
        \end{split}
    \]
    {\color{red} Origin article here is $e^{-s}h^\alpha h^{1-\alpha}$} we want to prove that the mapping \eqref{Chap2: eqn: 28} is differentiable. In view of Corollary \ref{Chap2: Coro: 17} we may as well prove that \eqref{Chap2: eqn: 28} is differentiable as a mapping into $\tilde{N}$. Now by the preceding lemma, the functions $f,g:S^\circ \mapsto \tilde{N}$ defined by $f(\alpha)=h^\alpha$ and $g(\alpha)=k^{1-\alpha}.$ are differentiable. It follows that for all $\alpha_0\in S^\circ$ we have 
    \[
        \begin{split}
            &\frac{1}{\alpha-\alpha_0}(f(\alpha)g(\alpha)-f(\alpha_0)g(\alpha_0))\\
            =&\frac{1}{\alpha-\alpha_0}f(\alpha)(g(\alpha)-g(\alpha_0))+\frac{1}{\alpha-\alpha_0}(f(\alpha)-f(\alpha_0))g(\alpha_0)\\
            \to & f(\alpha_0)g'(\alpha_0)+f'(\alpha_0)g(\alpha_0) \text{ as } \alpha\to \alpha_0
        \end{split}
    \]
so that also $f\cdot g: S^\circ \to \tilde{N}$ is differentiable.
\end{proof}
\begin{lemma}\label{Chap2: Lemma: 20}
    Let $t\in \mathbb{R}$ and put 
    \begin{equation}\label{Chap2: eqn: 29}
        \tilde{N}_{\frac{1}{2}+it}=\{a\in \tilde{N}|\forall s\in \mathbb{R}: \theta_s a=e^{-(\frac{1}{2}+it)}a\}.
    \end{equation}
    Let $a,b\in \tilde{N}_{\frac{1}{2}+it}$. Then $b^*a, ab^*\in L^1(M)$ and 
    \begin{equation}\label{Chap2: eqn: 30}
        \tr(b^*a)=\tr(ab^*).
    \end{equation}
\end{lemma}
\begin{proof}
    That $b^*a, ab^*\in L^1(M)$ follows from Definition \ref{Chap2: Def: 9} and \eqref{Chap2: eqn: 29}.\par
To prove \eqref{Chap2: eqn: 30}, suppose first that $a=b$. Then by Definition \ref{Chap2: Def: 13} and Lemma \ref{Chap2: Lemma: 5}
\[
    \tr(a^*a)=\tau(\chi_{]1,\infty[}(a^*a))=\tau(\chi_{]1,\infty[}(aa^*))=\tr(aa^*).
\]
In the general case, note that $a+ib\in \tilde{N}_{\frac{1}{2}+it}$ and 
\[
    b^*a=\frac{1}{4}\sum_{k=0}^3 i^k(a+i^kb)^*(a+i^kb)
\]
\[
    ab^*=\frac{1}{4}\sum_{k=0}^3 i^k(a+i^kb)(a+i^kb)^*
\]
The result follows since $\tr$ is linear.
\end{proof}
\begin{proposition}
Let $p,q\in [1,\infty]$ with $\frac{1}{p}+\frac{1}{q}=1$. Let $a\in L^p(M)$ and $b\in L^q(M)$. Then $ab,ba\in L^1(M)$ and 
\[
    \tr(ab) = \tr(ba).
\]
\end{proposition}
\begin{proof}
    If $p=1$ we have $a=h_\varphi$ for some $\varphi\in M_*$ and the result follows by Theorem \ref{Chap2: Thm: 7}:
    \[
        \tr(h_\varphi b)=\tr(h_{\varphi\cdot b})=(\varphi\cdot b)(1)=(b\cdot \varphi)(1)=\tr(h_{b\cdot\varphi})=\tr(bh_\varphi)
    \] 
    Now suppose that $p,q\in ]1,\infty[$. As usual, we easily see that $ab$ and $ba$ are in $L^1(M)$. By linearity, we may assume that $a\in L^p(M)_+$ and $b\in L^q(M)_+$. Now $a^p,b^q\in L^1(M)_+$ and by Lemma \ref{Chap2: Lemma: 19} the functions $F$ and $G$ on $S^\circ$ defined by $F(\alpha)=\tr(a^{p\alpha}b^{q(1-\alpha)})$ and $G(\alpha)=\tr(b^{q(1-\alpha)}a^{p\alpha})$ are analytic. For all $t\in \mathbb{R}$, we have $a^{p(\frac{1}{2}+it)}\in\tilde{N}_{\frac{1}{2}+it}$ and $b^{q(\frac{1}{2}+it)}\in \tilde{N}_{\frac{1}{2}+it}$ so that by Lemma \ref{Chap2: Lemma: 20}
    \[
        \begin{split}
            F(\frac{1}{2}+it)=&\tr(a^{p(\frac{1}{2}+it)}b^{q(\frac{1}{2}-it)})=\tr(a^{p(\frac{1}{2}+it)}(b^{q(\frac{1}{2}-it)})^*)\\
            =&\tr((b^{q(\frac{1}{2}-it)})^*a^{p(\frac{1}{2}+it)})=\tr(b^{q(\frac{1}{2}-it)}a^{p(\frac{1}{2}+it)})=G(\frac{1}{2}+it)
        \end{split}
    \] 
We conclude that $F=G$. In particular,
\[
    \tr(ab)=F(\frac{1}{p})=G(\frac{1}{p})=\tr(ba).
\]
\end{proof}
The proof of the next lemma is based on the 3 lines theorem for analytic functions (see e.g. [23, p.93]). The 3 lines theorem also holds for analytic functions $F$ with values in a Banach space (to see this, apply it to the scalar-valued functions $\alpha\mapsto v(F(\alpha))$, where $v$ is in the dual of the given Banach space).
\begin{lemma}\label{Chap2: Lemma: 22}
    Let $h,k\in L^1(M)_+$ and suppose that $\norm{h}_1=\norm{k}_1=1$. Then for all $\alpha\in S^\circ$, we have 
    \[
        \norm{h^\alpha k^{1-\alpha}}_1\leq 1
    \]
\end{lemma} 
\begin{proof}
    Write $s=\Re \alpha$, $t=\Im \alpha$. Then $h^s\in L^\frac{1}{s}(M)$ with $\norm{h^s}_{\frac{1}{s}}=1=s^{-s}\cdot s^s$, whence by Lemma \ref{Chap2: Lemma: 16} 
    \[
        h^s\in N(s^{-s},s).
    \]
    Similarly, 
    \[
        k^{1-s}\in N((1-s)^{-(1-s)},1-s).
    \]
    It follows that 
    \[
        \begin{split}
            h^s k^{1-s}\in& N(s^{-s},s)\cdot N((1-s)^{-(1-s)},1-s)\\
            \subset& N(s^{-s}(1-s)^{-(1-s)},s+(1-s))
        \end{split}
    \]
    whence also 
    \[
        h^\alpha k^{1-\alpha}=h^{it}h^s k^{1-s}k^{-it}\in N(s^{-s}(1-s)^{-(1-s)},1)
    \]
    Again by Lemma \ref{Chap2: Lemma: 16},
    \[
        \norm{h^\alpha k^{1-\alpha}}_1\leq s^{-s}(1-s)^{-(1-s)}
    \]
    Since $s\mapsto s^{-s}(1-s)^{-(1-s)}$ is bounded, the function $\alpha\mapsto h^\alpha k^{1-\alpha}:S^\circ \to L^1(M)$ is bounded. It is analytic by Lemma \ref{Chap2: Lemma: 19}. Hence we can apply the 3 lines theorem on each closed strip $\{a\in \mathbb{C}|\epsilon\leq \Re \alpha\leq 1-\epsilon\}$ and we obtain 
    \[
        \sup_{t\leq \Re \alpha\leq 1-\epsilon}\norm{h^\alpha k^{1-\alpha}}_1\leq \epsilon^{-\epsilon}(1-\epsilon)^{-(1-\epsilon)}.
    \]
    Hence for fixed $a\in S^\circ$, the inequality 
    \[
        \norm{h^\alpha k^{1-\alpha}}_1\leq \epsilon^{-\epsilon}(1-\epsilon)^{-(1-\epsilon)}
    \]
    holds for all $\epsilon\in \mathbb{R}_+$ such that $\epsilon\leq \Re \alpha\leq 1-\epsilon$. Since 
    \[
        \epsilon^{-\epsilon}(1-\epsilon)^{-(1-\epsilon)}=e^{-\epsilon\log \epsilon}e^{-(1-\epsilon)\log (1-\epsilon)}\to 1 \text{ as } \epsilon\to 0,
    \]
    it follows that 
    \[
        \norm{h^\alpha k^{1-\alpha}}_1\leq 1 
    \]
    This proves the lemma.
\end{proof}
\begin{theorem}[H\"older's inequality]
    Let $p,q\in [1,\infty]$ with $\frac{1}{p}+\frac{1}{q}=1$. Let $a\in L^p(M)$ and $b\in L^q(M)$. Then  
    \[
        \norm{ab}_1\leq \norm{a}_p\norm{b}_q.
    \]
\end{theorem}
\begin{proof}
    If $p=1$, we have $a=h_\varphi$ for some $\varphi\in M_*$ and 
\[
    \norm{h_\varphi b}_1=\norm{h_{\varphi\cdot b}}_1=\norm{\varphi\cdot b}\leq \norm{\varphi}\norm{b}_\infty=\norm{h_\varphi}_1\cdot \norm{b}_\infty
\]
for all $b\in L^\infty(M)=M$. The case $q=1$ is quite similar to this.\par
Now assume $p,q\in ]1,\infty[$, and $\norm{a}_p=1$, $\norm{b}_q=1$. Let $a=u\abs{a}$ be the (usual) polar decomposition of $a$ and $b=\abs{b^*}v$ the right polar decomposition of $b$. Then $\abs{a}^p, \abs{b^*}^q\in L^1(M)$ with $\norm{\abs{a}^p}=\norm{\abs{b^*}_r^q}_1=1$ and Lemma \ref{Chap2: Lemma: 22} applies:
\[
    \begin{split}
        \norm{ab}_1=&\norm{u\abs{a}\abs{b^*}v}_1\leq \norm{\abs{a}\abs{b^*}}_1\\
        =&\norm{\abs{a}^\frac{p}{p}\abs{b^*}^\frac{q}{q}}_1\leq 1.
    \end{split}
\]
\end{proof}
\begin{proposition}\label{Chap2: Prop: 24}
    Let $p,q\in [1,\infty]$ with $\frac{1}{p}+\frac{1}{q}=1$. Let $a\in L^p(M)$. Then 
    \[
        \norm{a}_p=\sup\{\abs{\tr(ab)}|b\in L^q(M),\norm{b}_q\leq 1\}.
    \]
\end{proposition}
\begin{proof}
    If $p=1$ or $p=\infty$ this is well-known (since $\tr(ch_\varphi)=\tr(h_\varphi c)=\varphi(c)$ for all $\varphi\in M_*$ and $c\in M$). Suppose that $1<p<\infty$. We may assume that $\norm{a}_p=1$. Then putting $b=\abs{a}^\frac{p}{q}u^*$, where $a=u\abs{a}$ is the polar decomposition of $a$, we have $b\in L^q(M)$ with $\norm{b}_q=\norm{\abs{a}^\frac{p}{q}u^*}_q=\tr(\abs{a}^p)^\frac{1}{q}=1$ and  
    \[
        \tr(ab)=\tr(u\abs{a}\abs{a}^\frac{p}{q}u^*)=\tr(\abs{a}^p)=1.
    \]  
Hence 
\[
   \norm{a}_p=1\leq \sup\{\abs{\tr(ab)}|b\in L^q(M),\norm{b}_q\leq 1\}.
\]
The converse inequality follows from H\"older's inequality (together with \eqref{Chap2: eqn: 26}). 
\end{proof}
\begin{corollary}
    $\norm{\cdot}_p$ is a norm on $L^p(M)$.
\end{corollary}
\begin{proof}
    The inequality 
\[
    \norm{a+b}_p\leq \norm{a}_p+\norm{b}_p
\]
    follows immediately from Proposition \ref{Chap2: Prop: 24}.
\end{proof}
\begin{proposition}
    On $L^p(M)$, the norm topology is exactly the topology induced from $\tilde{N}$. 
\end{proposition}
\begin{proof}
    Now that we know that $\norm{\cdot}_p$ is a norm, this is a corollary of Lemma \ref{Chap2: Lemma: 16}.
\end{proof}
\begin{corollary}\label{Chap2: Coro: 27}
    $(L^p(M),\norm{\cdot}_p)$ is a Banach space. 
\end{corollary}
\begin{proof}
    From the definition of $L^p(M)$ it follows that it is a closed subspace of the complete topological vector space $\tilde{N}$. Hence it is complete for the uniform structure induced from $\tilde{N}$. By Lemma \ref{Chap2: Lemma: 16}, this is simply the uniform structure coming from the norm. Hence $L^p(M)$ is a complete normed space.
\end{proof}
\begin{corollary}
    $(L^2(M),\norm{\cdot}_2)$ is a Hilbert space with the inner product 
    \[
        (a|b)_{L^2(M)}=\tr(b^*a)\quad (=\tr(ab^*)),a,b\in L^2(M).
    \]
\end{corollary}
\begin{proof}
    That $(a,b)\mapsto (a|b)_{L^2(M)}$ is an inner product defining the norm $\norm{\cdot}_2$ is easily verified. By Corollary \ref{Chap2: Coro: 27}, $L^2(M)$ is complete.
\end{proof}
\begin{remark}
    Let $t\in \mathbb{R}$. Define $\tilde{N}_{\frac{1}{2}+it}$ as in Lemma \ref{Chap2: Lemma: 20}. Then 
    \[
        (a,b)\mapsto \tr(b^*a)
    \]
is an inner product on $\tilde{N}_{\frac{1}{2}+it}$ and 
\[
    a\mapsto \tr(a^*a)^\frac{1}{2}
\]
is a norm which we shall denote by $\norm{\cdot}_2$ (as in the case $t=0$ where $\tilde{N}_{\frac{1}{2}}=L^2(M)$). Note that 
\[
    \abs{\tr(b^*a)}\leq \norm{a}_2\norm{b}_2
\]
and 
\[
    \norm{a+b}_2^2+\norm{a-b}_2^2=2\norm{a}_2^2+2\norm{b}_2^2
\]
for all $a,b\in \tilde{N}_{\frac{1}{2}+it}$.
\end{remark}
% \end{document}